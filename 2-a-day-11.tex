\documentclass[12pt]{article}
% math symbols
\usepackage{amssymb,amsmath}
% for different compilers
\usepackage{ifpdf}
% geometry of page
\usepackage[left=1.8cm,right=1.8cm,top=1cm,bottom=1cm]{geometry}
% float pictures
\usepackage{wrapfig}

% if pdflatex, then
\ifpdf
 \usepackage[english,russian]{babel}
 \usepackage[utf8]{inputenc}
 \usepackage[unicode]{hyperref}
 \usepackage[pdftex]{graphicx}
 \usepackage{cmlgc}
% if xelatex, then
\else
% math fonts
 \usepackage{fouriernc}
% xelatex specific packages
 \usepackage[xetex]{hyperref}
 \usepackage{xunicode}	% some extra unicode support
 \usepackage{xltxtra}	% \XeLaTeX macro
 \defaultfontfeatures{Mapping=tex-text}
 \usepackage{polyglossia}	% instead of babel in xelatex
 \setdefaultlanguage{russian}
% fonts
 \setromanfont{Charis SIL}
 \setsansfont{OfficinaSansC} 
 \setmonofont{Consolas}
\fi

% several pictures in one figure
\usepackage{subfig}
% calc in TeX expressions
\usepackage{calc}
% nice pictures and plots
\usepackage{pgfplots,tikz,circuitikz}
% different libraries for pictures
\usetikzlibrary{%
  arrows,%
  calc,%
  patterns,%
  decorations.pathreplacing,%
  decorations.pathmorphing,%
  decorations.markings%
}
\tikzset{>=latex}

% colors of the hyperlinks
\hypersetup{colorlinks,%
  citecolor=blue,%
  urlcolor=blue,%
  linkcolor=red
}

\tolerance=1000
\emergencystretch=0.74cm

% счётчик задач
\newcounter{notask}
\setcounter{notask}{1}

% условие без картинки
\newcommand{\task}[1]{
  \hrule
  \hbox to \textwidth {%
    \vrule
    \parbox[t]{0.04\textwidth}{\smallskip \centering \arabic{notask}}%
    \vrule%
    \hfill%
    \parbox[t]{0.93\textwidth}{\smallskip #1 \smallskip}\hfill%
    \vrule
  }
  \hrule
  \addtocounter{notask}{1}
  \pagebreak[2]
}

\newlength{\h}
\newsavebox{\taskbox}
\newlength{\x}
\newsavebox{\pictbox}

% условие с картинкой (картинка выравнивается по центру)
\newcommand{\taskpic}[2]{
  \savebox{\taskbox}{\parbox[t]{0.93\textwidth-4.3cm}{\smallskip #1 \smallskip}}
  \savebox{\pictbox}{\parbox[t]{4cm}{\smallskip \centering
      \vspace{0pt} #2 \smallskip}}
  \h=\ht\taskbox
  \advance\h\dp\taskbox
  \x=\ht\pictbox
  \advance\x\dp\pictbox
  \hrule
  \hbox to \textwidth {%
    \vrule\parbox[t][\maxof{\h}{\x}][t]{0.04\textwidth}{ \smallskip
      \centering \arabic{notask} }\vrule%
    \hfill\parbox[t][\maxof{\h}{\x}][t]{0.93\textwidth-4.3cm}{\smallskip #1
        \smallskip}\hfill\vrule%
    \hfill\parbox[t][\maxof{\h}{\x}][c]{4cm}{\hfil #2 \hfil}\hfill\vrule
  }
  \hrule
  \addtocounter{notask}{1}
  \pagebreak[2]
}

\newcommand{\com}[1]{{\Large{\texttt{{\color{red}(#1)}}}}}
% русские единицы измерения в формулах
\newcommand{\unit}[1]{\text{ #1}}
\newcommand{\eps}{\varepsilon}
% не писать номер рисунка в маленьких картинках
\renewcommand{\thesubfigure}{\relax}

% настройки для tikz
% interface = лохматый отрезок
% spring = пружинка
% >=latex --- тип стрелочки
\tikzset{>=latex,%
  interface/.style={postaction={draw,decorate,decoration={border,angle=45,amplitude=0.2cm,segment
        length=1.4323mm}}},%
  spring/.style={decorate,decoration={snake,amplitude=1mm, segment length=2mm},thick}}

\pagestyle{empty}


\begin{document}

\taskpic{Из среды с показателем преломления $n_0$ в неоднородную среду 
с показателем преломления $n = n_0 \sqrt{1-\frac yH}$ под углом $\varphi_0$ входит 
луч света. На какую максимальную глубину сможет проникнуть луч? 
При каком значении угла падения $\varphi_0$ расстояние между точками 
входа и выхода луча максимально?}{
\begin{tikzpicture}
  \fill[gray!10] (0.2,2) rectangle ++(3.6,-2);
  \draw[thick] (0.2,2) -- ++(3.6,0);
  \draw[dashed,thick,->] (2,3.5) -- ++(0,-3) node [right] {$y$};
  \draw[very
  thick,draw=red,postaction=decorate,decoration={markings,mark=at
    position .55 with {\arrow[red]{>}}}] (1,3.5) -- (2,2);
  \draw[blue] (2,2.5) arc (90:90+atan(2/3):0.5);
  \draw[blue] (1.8,2.7) node {$\alpha$};
  \draw (0.5,3) node {$n_0$};
  \draw (0.7,1) node {$n(y)$};
\end{tikzpicture}
}

\task{На какую минимальную высоту надо поднять поршень, лежащий на 
поверхности воды в герметичном сосуде, чтобы вся вода в нем 
испарилась? Толщина слоя воды от дна сосуда --- $h$, плотность воды 
--- $\rho$, молярная масса --- $\mu$, давление насыщенного водяного пара 
$p$. Температура $T$ воды и пара в сосуде поддерживается постоянной.}

\vspace{1cm}

\task{Тонкая доска массой $m_1$ и длиной $L$ скользит по гладкому столу 
со скоростью $v_1$. Маленькая шайба массой $m_2$ плавно въезжает на 
доску со скоростью $v_2$ относительно Земли. При каких значениях 
коэффициента трения между доской и шайбой потери механической 
энергии при их взаимодействии максимальны?}

\task{Груз поднимают при помощи невесомого поршня, скользящего без 
трения в вертикальном теплоизолированном цилиндре. Под поршнем 
находится идеальный одноатомный газ, медленно нагреваемый при 
помощи электронагревателя с КПД, равным $\eta = 1/2$. Определить КПД 
подъемного устройства, если атмосферное давление отсутствует.}

\vspace{1cm}

\taskpic{В точках A и B жесткого невесомого стержня укреплены два 
маленьких шарика. В точке O стержень закреплен и может свободно 
вращаться в вертикальной плоскости. В начальный момент времени 
стержень отклоняют от вертикального положения на очень 
маленький угол и отпускают. Найти силу, действующую на шарик В со 
стороны стержня в момент, когда угол между стержнем и вертикалью 
равен $\alpha$. Масса каждого груза $m$, длина стержня $L$, 
OA=AB.}{
\begin{tikzpicture}
  \draw[thick] (2,2) circle (1.5);
  \draw[very thick] (2,3.5) -- (2,2) -- ++(-135:1.5);
  \filldraw[black] (2,2.75) circle (0.1) node[right] {$A$};
  \filldraw[black] (2,3.5) circle (0.1) node[above right] {$B$};
  \draw (2,2) node[right] {$O$};
  \filldraw[black] (2,2) ++(-135:0.75) circle (0.1);
  \filldraw[black] (2,2) ++(-135:1.5) circle (0.1);

  \draw[dashed,blue] (2,2) -- ++(0,-1.5);
  \draw[blue] (2,2) ++ (0,-0.5) arc (-90:-135:0.5);
  \draw[blue] (1.75,1.25) node {$\alpha$};
\end{tikzpicture}
}

\task{Протон (ядро атома водорода) и альфа-частица (ядро атома гелия, 
состоящее из двух протонов и двух нейтронов) разгоняются 
одинаковой ускоряющей разностью потенциалов и влетают в 
однородное магнитное поле перпендикулярно его линиям. 
Определить отношение радиусов орбит и нарисовать траектории 
движения частиц в магнитном поле. Массы протона и нейтрона равны.}

\vspace{1cm}

\taskpic{Верхняя поверхность большой плоской пластины 
поддерживается при постоянной температуре $T_1$, а нижняя --- при 
температуре $T_2$ ($T_2>T_1$). Оцените подъемную силу 1~м$^2$ такой 
пластины, если она находится в атмосфере разреженного газа с 
давлением $p_0$ и температурой $T_0$ ($T_1<T_0<T_2$).}{
\begin{tikzpicture}
  \draw[very thick] (0.2,2) rectangle ++(3.6,-0.3);
  \draw (1,2.3) node {$T_1$};
  \draw (1,1.4) node {$T_2$};
  \draw[very thick,->] (2,2) -- ++(0,1) node[right] {$F$};
\end{tikzpicture}
}

\task{На одном из островов Бермудского треугольника ускорение 
свободного падения отклонено на юг и составляет угол $\alpha$ с 
вертикалью. На каком расстоянии от туземца упадет стрела, 
выпущенная им вертикально вверх с начальной скоростью $v_0$? В 
каком направлении следует выпустить стрелу для того, чтобы она 
вернулась обратно?}

\vspace{1cm}

\taskpic{Равнобедренный треугольник состоит из 3 маленьких шариков, 
скрепленных невесомыми жесткими стержнями. Заряд верхнего 
шарика $2\alpha q$, а заряды каждого из нижних шариков равны $(1-\alpha)q$. 
Массы шариков одинаковы. Эта конструкция находится в равновесии 
в сонаправленных гравитационном поле $g$ и электростатическом 
поле $E$. Определить устойчивость равновесия в зависимости от 
параметра $\alpha$.}{
\begin{tikzpicture}
  \coordinate (a) at (1,1);
  \coordinate (b) at (2,3.5);
  \coordinate (c) at (3,1);
  \draw[very thick] (a) -- (b) -- (c) -- cycle;
  \filldraw[black] (a) circle (0.1) node [below] {$(1-\alpha) q$};
  \filldraw[black] (b) circle (0.1) node [above] {$2\alpha q$};
  \filldraw[black] (c) circle (0.1) node [below] {$(1-\alpha) q$};
  \draw[thick,->] (0.5,3.5) -- ++(0,-1) node[midway,right] {$\vec{g}$};
  \draw[thick,->] (3.5,3.5) -- ++(0,-1) node[midway,left] {$\vec{E}$};
\end{tikzpicture}
}

\taskpic{Две частицы, имеющие заряды $q_1$ и $q_2$ и равные массы, могут 
скользить без трения вдоль двух параллельных прямых, 
расположенных на расстоянии $L$. В начальный момент частица $q_1$ 
движется со скоростью $v_0$ из бесконечности, приближаясь к 
покоящейся частице $q_2$. Определить установившиеся скорости 
частиц.}{
\begin{tikzpicture}
  \draw[very thick] (0.2,3) -- +(3.6,0);
  \draw[very thick] (0.2,1) -- +(3.6,0);
  \draw[thick,blue,<->] (2,3) -- +(0,-2) node[midway,right] {$L$};
  \filldraw[black] (0.75,3) circle (0.15) node [below=5] {$q_1,m$};
  \filldraw[black] (3.25,1) circle (0.15) node [below=5] {$q_2,m$};
  \draw[very thick,->] (0.75,3.25) -- +(1,0) node [midway,above] {$v_0$};
\end{tikzpicture}
}

\vspace{1cm}

\task{Две частицы движутся вдоль одной прямой навстречу друг другу 
со скоростями $v_1$ и $v_2$ соответственно. После их абсолютно 
неупругого столкновения скорости частиц равны $v$. Найти 
отношение масс частиц.}

\taskpic{Конец однородного стержня длиной $L$ и массой $M$ закреплен на 
шарнире так, что стержень может вращаться в любом направлении 
без трения. На расстоянии $l < L$ от точки закрепления к стержню 
прикреплена пружина жесткости $k$, длина которой в 
недеформированном состоянии пренебрежимо мала. В какой точке 
пространства следует закрепить другой конец пружины для того, 
чтобы стержень находился в положении безразличного 
равновесия?}{
\begin{tikzpicture}
   \draw[very thick] (0.5,0.5) -- ++(3,2.5);
   \coordinate (a) at ($(0.5,0.5)!0.6!(3.5,3)$);
   \filldraw[black] (a) circle (0.03);
   \filldraw[brown] (0.5,0.5) circle (0.05);
   \draw[spring] (1,3) node [left] {?}-- (a);
   \draw[blue!80,<->] (0.5,0.5) ++(0.1,-0.1) -- ($(a)+(0.1,-0.1)$)
   node[midway,below] {$l$};
   \draw[blue!80,<->] ($(a)+(0.1,-0.1)$) -- (3.6,2.9) node[below=20,right=-20] {$L-l$};
\end{tikzpicture}
}

\vspace{1cm}

\taskpic{Цилиндрический сосуд с массивным поршнем находится в лифте. 
Под поршнем находится идеальный газ, давление которого в $k$ раз 
отличается от атмосферного. Первоначально система находится в 
равновесии. Расстояние от дна сосуда до поршня равно $h$. Найти 
расстояние между поршнем и дном сосуда в лифте, двигающимся 
вверх с ускорением $a$. Температуру газа считать постоянной, 
трением между поршнем и стенками сосуда 
пренебречь.}{
\begin{tikzpicture}
   \draw[thick] (1,4) -- ++(0,-3.5) -- ++(2,0) -- ++(0,3.5);
   \draw[thick,pattern=north east lines] (1,2.5) rectangle ++(2,0.25);
   \draw[line width=2] (2,2.75) -- ++(0,1.25);
   \draw[pattern=dots,pattern color=green] (1,2.5) rectangle ++(2,-2);
   \draw[dashed,blue] (1,0.5) -- +(-0.5,0) node [coordinate, near end] (a) {};
   \draw[dashed,blue] (1,2.5) -- +(-0.5,0) node [coordinate, near end] (b) {};
   \draw[blue,|<->|] (a) -- node[fill=white] {$h$} (b);
   \draw[very thick,->] (3.5,1.5) -- ++(0,1) node [above] {$a$};
\end{tikzpicture}
}

\taskpic{Идеальную пружину нулевой начальной длины, один конец 
которой закреплен, а к другому концу подвешен точечный груз 
массы $M$, растягивают до длины $L$ и отводят так, что угол с 
горизонталью составляет 45$^\circ$. Определить форму и длину 
траектории груза. Жесткость пружины равна $k$, ускорение 
свободного падения $g$.}{
\begin{tikzpicture}
  \draw[blue, dashed,thick] (0.5,4) -- (4,4);
  \draw[blue] (2,4) arc (0:-45:1);
  \draw[blue] (2.7,3.5) node {$\alpha = 45^\circ$};
  \draw[spring] (1,4) -- ++(2,-2) node [midway, below=7] {$L$};
  \filldraw[black] (3,2) circle (0.1) node[right] {$M$};
  \filldraw[brown] (1,4) circle (0.05);
\end{tikzpicture}
}

\vspace{1cm}

\taskpic{На гладком конусе с углом при вершине 120$^\circ$ шарнирно 
закреплен невесомый нерастяжимый стержень длиной $L$. К концу 
стержня прикреплен груз. Вся система вращается вокруг 
вертикальной оси. При какой частоте вращения груз разорвет 
стержень, если стержень выдерживает утроенный вес 
груза?}{
\begin{tikzpicture}
  \draw[very thick] ++(0.25,1) -- ++(30:2cm) --++(-30:2cm) -- cycle;
  \draw[thick,dashed] (2,3) -- ++(0,-2.5);
  \draw (2,2) -- ++(0,0.25) node (b) {} -- ++(210:1cm) node (a) {};
  \filldraw[brown] (b) circle (0.05);
  \shadedraw[shading=ball] (a) circle (0.2);
  \draw[blue] ($(a)+(0.4,0.6)$) node {$L$};
  \draw[thick,->] (2.3,3) arc (0:-180:0.3);
\end{tikzpicture}
}

\task{Сопротивление проволоки изменяется с температурой по закону 
$R = R_0(1 + \alpha t)$, где $R_0$ --- сопротивление при температуре, равной 
0$^\circ$C. Как должно изменяться подводимое к проволоке напряжение 
для того, чтобы температура проволоки линейно росла со временем? 
Теплоемкость проволоки равна $C$, отвода тепла нет.}

\vspace{1cm}

\taskpic{Заряженная частица $(q, m)$ может скользить без трения по 
проволочному кольцу радиусом $R$, расположенному вертикально. 
Какое вертикальное электрическое поле нужно приложить, чтобы 
частота малых колебаний частицы уменьшилась в два 
раза?}{
\begin{tikzpicture}
  \draw[very thick] (2.5,2) circle (1.5);
  \filldraw[black] (2.5,0.5) circle (0.15) node[below=5] {$q,m$};
  \draw[thick,->] (0.5,3) -- +(0,-1) node [midway,right] {$g$};
\end{tikzpicture}
}

\taskpic{В вертикальную трубу с бесконечными стенками поместили 
цилиндрическую капсулу. Сила трения между капсулой и стенками 
трубы прямо пропорциональна относительной скорости 
соприкасающихся поверхностей. Капсуле придали начальную 
линейную скорость, направленную вверх, и начальную угловую 
скорость. Когда капсула опустилась на начальную высоту, модуль 
линейной скорости изменился на $v$ относительно модуля начальной 
скорости, а угловая скорость стала равна $\omega$. При дальнейшем 
спуске капсула повернулась на угол $\alpha$ (на бесконечности). Время 
подъема от начальной высоты до наивысшей точки отличалось от 
времени спуска до начальной высоты на $T$. До какой максимальной 
высоты $H$ поднялась капсула, относительно начальной высоты, если 
радиус капсулы $R$, а ее масса распределена по боковой 
поверхности?}{
\begin{tikzpicture}[line join=round]
\filldraw[fill=white,draw=gray](0,-2.41)--(-.212,-2.398)--(-.404,-2.362)--(-.556,-2.305)--(-.653,-2.234)--(-.687,-2.155)--(-.653,-2.076)--(-.556,-2.005)--(-.404,-1.949)--(-.212,-1.912)--(0,-1.9)--(.212,-1.912)--(.404,-1.949)--(.556,-2.005)--(.653,-2.076)--(.687,-2.155)--(.653,-2.234)--(.556,-2.305)--(.404,-2.362)--(.212,-2.398)--cycle;
\filldraw[fill=blue!80](-.215,-.185)--(-.215,.677)--(0,.69)--(0,-.172)--cycle;
\filldraw[fill=blue!80,draw=none](0,-.176)--(-.029,-.174)--(0,-.172)--(.029,-.174)--cycle;
\draw(-.029,-.174)--(0,-.172)--(.029,-.174);
\filldraw[fill=blue!80](0,-.172)--(0,.69)--(.215,.677)--(.215,-.185)--cycle;
\filldraw[fill=blue!80,draw=none](.212,-.188)--(0,-.176)--(.029,-.174)--(.186,-.183)--cycle;
\draw(.029,-.174)--(.186,-.183);
\filldraw[fill=blue!80,draw=none](.212,-.188)--(.186,-.183)--(.215,-.185)--(.242,-.19)--cycle;
\draw(.186,-.183)--(.215,-.185)--(.242,-.19);
\filldraw[fill=blue!80,draw=none](0,-.176)--(-.212,-.188)--(-.186,-.183)--(-.029,-.174)--cycle;
\draw(-.186,-.183)--(-.029,-.174);
\filldraw[fill=blue!80,draw=none](-.212,-.188)--(-.242,-.19)--(-.215,-.185)--(-.186,-.183)--cycle;
\draw(-.242,-.19)--(-.215,-.185)--(-.186,-.183);
\filldraw[fill=blue!80,draw=none](0,.686)--(.029,.688)--(0,.69)--(-.029,.688)--cycle;
\draw(.029,.688)--(0,.69)--(-.029,.688);
\filldraw[fill=blue!80,draw=none](-.212,.674)--(0,.686)--(-.029,.688)--(-.186,.679)--cycle;
\draw(-.029,.688)--(-.186,.679);
\filldraw[fill=blue!80](-.409,-.222)--(-.409,.64)--(-.215,.677)--(-.215,-.185)--cycle;
\filldraw[fill=blue!80,draw=none](-.212,.674)--(-.186,.679)--(-.215,.677)--(-.242,.672)--cycle;
\draw(-.186,.679)--(-.215,.677)--(-.242,.672);
\filldraw[draw=gray,fill opacity=0.2,fill=blue!20](-.212,-1.912)--(-.212,2.398)--(0,2.41)--(0,-1.9)--cycle;
\filldraw[fill=blue!80,draw=none](0,.686)--(.212,.674)--(.186,.679)--(.029,.688)--cycle;
\draw(.186,.679)--(.029,.688);
\filldraw[fill=blue!80](.215,-.185)--(.215,.677)--(.409,.64)--(.409,-.222)--cycle;
\filldraw[fill=blue!80,draw=none](.212,.674)--(.242,.672)--(.215,.677)--(.186,.679)--cycle;
\draw(.242,.672)--(.215,.677)--(.186,.679);
\filldraw[draw=gray,fill opacity=0.2,fill=blue!20](0,-1.9)--(0,2.41)--(.212,2.398)--(.212,-1.912)--cycle;
\filldraw[fill=blue!80,draw=none](.404,-.225)--(.212,-.188)--(.242,-.19)--(.383,-.217)--cycle;
\draw(.242,-.19)--(.383,-.217);
\filldraw[fill=blue!80,draw=none](.404,-.225)--(.383,-.217)--(.409,-.222)--(.43,-.23)--cycle;
\draw(.383,-.217)--(.409,-.222)--(.43,-.23);
\filldraw[fill=blue!80,draw=none](.404,.637)--(.383,.645)--(.242,.672)--(.212,.674)--cycle;
\draw(.383,.645)--(.242,.672);
\filldraw[fill=blue!80](.409,-.222)--(.409,.64)--(.563,.583)--(.563,-.279)--cycle;
\filldraw[fill=blue!80,draw=none](.404,.637)--(.43,.632)--(.409,.64)--(.383,.645)--cycle;
\draw(.43,.632)--(.409,.64)--(.383,.645);
\filldraw[draw=gray,fill opacity=0.2,fill=blue!20](.212,-1.912)--(.212,2.398)--(.404,2.362)--(.404,-1.949)--cycle;
\filldraw[fill=blue!80,draw=none](-.404,-.225)--(-.383,-.217)--(-.242,-.19)--(-.212,-.188)--cycle;
\draw(-.383,-.217)--(-.242,-.19);
\filldraw[fill=blue!80,draw=none](-.404,-.225)--(-.43,-.23)--(-.409,-.222)--(-.383,-.217)--cycle;
\draw(-.43,-.23)--(-.409,-.222)--(-.383,-.217);
\filldraw[fill=blue!80,draw=none](-.404,.637)--(-.212,.674)--(-.242,.672)--(-.383,.645)--cycle;
\draw(-.242,.672)--(-.383,.645);
\filldraw[fill=blue!80](-.563,-.279)--(-.563,.583)--(-.409,.64)--(-.409,-.222)--cycle;
\filldraw[fill=blue!80,draw=none](-.404,.637)--(-.383,.645)--(-.409,.64)--(-.43,.632)--cycle;
\draw(-.383,.645)--(-.409,.64)--(-.43,.632);
\filldraw[draw=gray,fill opacity=0.2,fill=blue!20](-.404,-1.949)--(-.404,2.362)--(-.212,2.398)--(-.212,-1.912)--cycle;
\filldraw[fill=blue!80,draw=none](-.556,-.281)--(-.542,-.271)--(-.43,-.23)--(-.404,-.225)--cycle;
\draw(-.542,-.271)--(-.43,-.23);
\filldraw[fill=blue!80,draw=none](-.556,-.281)--(-.577,-.289)--(-.563,-.279)--(-.542,-.271)--cycle;
\draw(-.577,-.289)--(-.563,-.279)--(-.542,-.271);
\filldraw[fill=blue!80,draw=none](-.556,.581)--(-.404,.637)--(-.43,.632)--(-.542,.591)--cycle;
\draw(-.43,.632)--(-.542,.591);
\filldraw[fill=blue!80](-.662,-.351)--(-.662,.511)--(-.563,.583)--(-.563,-.279)--cycle;
\filldraw[fill=blue!80,draw=none](-.556,.581)--(-.542,.591)--(-.563,.583)--(-.577,.573)--cycle;
\draw(-.542,.591)--(-.563,.583)--(-.577,.573);
\filldraw[draw=gray,fill opacity=0.2,fill=blue!20](-.556,-2.005)--(-.556,2.305)--(-.404,2.362)--(-.404,-1.949)--cycle;
\filldraw[fill=blue!80,draw=none](.556,-.281)--(.404,-.225)--(.43,-.23)--(.542,-.271)--cycle;
\draw(.43,-.23)--(.542,-.271);
\filldraw[fill=blue!80,draw=none](.556,-.281)--(.542,-.271)--(.563,-.279)--(.577,-.289)--cycle;
\draw(.542,-.271)--(.563,-.279)--(.577,-.289);
\filldraw[fill=blue!80,draw=none](.556,.581)--(.542,.591)--(.43,.632)--(.404,.637)--cycle;
\draw(.542,.591)--(.43,.632);
\filldraw[fill=blue!80](.563,-.279)--(.563,.583)--(.662,.511)--(.662,-.351)--cycle;
\filldraw[fill=blue!80,draw=none](.556,.581)--(.577,.573)--(.563,.583)--(.542,.591)--cycle;
\draw(.577,.573)--(.563,.583)--(.542,.591);
\filldraw[draw=gray,fill opacity=0.2,fill=blue!20](.404,-1.949)--(.404,2.362)--(.556,2.305)--(.556,-2.005)--cycle;
\filldraw[fill=blue!80,draw=none](.653,-.352)--(.556,-.281)--(.577,-.289)--(.649,-.341)--cycle;
\draw(.577,-.289)--(.649,-.341);
\filldraw[fill=blue!80,draw=none](.653,-.352)--(.649,-.341)--(.662,-.351)--(.667,-.362)--cycle;
\draw(.649,-.341)--(.662,-.351)--(.667,-.362);
\filldraw[fill=blue!80,draw=none](.653,.51)--(.649,.521)--(.577,.573)--(.556,.581)--cycle;
\draw(.649,.521)--(.577,.573);
\filldraw[fill=blue!80](.662,-.351)--(.662,.511)--(.696,.431)--(.696,-.431)--cycle;
\filldraw[fill=blue!80,draw=none](.653,.51)--(.667,.5)--(.662,.511)--(.649,.521)--cycle;
\draw(.667,.5)--(.662,.511)--(.649,.521);
\filldraw[draw=gray,fill opacity=0.2,fill=blue!20](.556,-2.005)--(.556,2.305)--(.653,2.234)--(.653,-2.076)--cycle;
\filldraw[fill=blue!80,draw=none](-.653,-.352)--(-.649,-.341)--(-.577,-.289)--(-.556,-.281)--cycle;
\draw(-.649,-.341)--(-.577,-.289);
\filldraw[fill=blue!80,draw=none](-.653,-.352)--(-.667,-.362)--(-.662,-.351)--(-.649,-.341)--cycle;
\draw(-.667,-.362)--(-.662,-.351)--(-.649,-.341);
\filldraw[fill=blue!80,draw=none](-.653,.51)--(-.556,.581)--(-.577,.573)--(-.649,.521)--cycle;
\draw(-.577,.573)--(-.649,.521);
\filldraw[fill=blue!80](-.696,-.431)--(-.696,.431)--(-.662,.511)--(-.662,-.351)--cycle;
\filldraw[fill=blue!80,draw=none](-.653,.51)--(-.649,.521)--(-.662,.511)--(-.667,.5)--cycle;
\draw(-.649,.521)--(-.662,.511)--(-.667,.5);
\filldraw[draw=gray,fill opacity=0.2,fill=blue!20](-.653,-2.076)--(-.653,2.234)--(-.556,2.305)--(-.556,-2.005)--cycle;
\filldraw[fill=blue!80,draw=none](-.687,-.431)--(-.692,-.42)--(-.667,-.362)--(-.653,-.352)--cycle;
\draw(-.692,-.42)--(-.667,-.362);
\filldraw[fill=blue!80,draw=none](-.687,.431)--(-.653,.51)--(-.667,.5)--(-.692,.442)--cycle;
\draw(-.667,.5)--(-.692,.442);
\filldraw[draw=gray,fill opacity=0.2,fill=blue!20](-.687,-2.155)--(-.687,2.155)--(-.653,2.234)--(-.653,-2.076)--cycle;
\filldraw[fill=blue!80,draw=none](.687,-.431)--(.653,-.352)--(.667,-.362)--(.692,-.42)--cycle;
\draw(.667,-.362)--(.692,-.42);
\filldraw[fill=blue!80,draw=none](.687,.431)--(.692,.442)--(.667,.5)--(.653,.51)--cycle;
\draw(.692,.442)--(.667,.5);
\filldraw[draw=gray,fill opacity=0.2,fill=blue!20](.653,-2.076)--(.653,2.234)--(.687,2.155)--(.687,-2.155)--cycle;
\filldraw[fill=blue!80,draw=none](.212,-.674)--(0,-.686)--(-.212,-.674)--(-.404,-.637)--(-.556,-.581)--(-.653,-.51)--(-.687,-.431)--(-.653,-.352)--(-.556,-.281)--(-.404,-.225)--(-.212,-.188)--(0,-.176)--(.212,-.188)--(.404,-.225)--(.556,-.281)--(.653,-.352)--(.687,-.431)--(.653,-.51)--(.556,-.581)--(.404,-.637)--cycle;
\filldraw[fill=blue!80,draw=none](0,.176)--(.212,.188)--(.404,.225)--(.556,.281)--(.653,.352)--(.687,.431)--(.653,.51)--(.556,.581)--(.404,.637)--(.212,.674)--(0,.686)--(-.212,.674)--(-.404,.637)--(-.556,.581)--(-.653,.51)--(-.687,.431)--(-.653,.352)--(-.556,.281)--(-.404,.225)--(-.212,.188)--cycle;
\filldraw[draw=gray,fill opacity=0.2,fill=blue!20](.687,-2.155)--(.687,2.155)--(.653,2.076)--(.653,-2.234)--cycle;
\filldraw[draw=gray,fill opacity=0.2,fill=blue!20](-.653,-2.234)--(-.653,2.076)--(-.687,2.155)--(-.687,-2.155)--cycle;
\filldraw[draw=gray,fill opacity=0.2,fill=blue!20](-.556,-2.305)--(-.556,2.005)--(-.653,2.076)--(-.653,-2.234)--cycle;
\filldraw[draw=gray,fill opacity=0.2,fill=blue!20](.653,-2.234)--(.653,2.076)--(.556,2.005)--(.556,-2.305)--cycle;
\filldraw[draw=gray,fill opacity=0.2,fill=blue!20](-.404,-2.362)--(-.404,1.949)--(-.556,2.005)--(-.556,-2.305)--cycle;
\filldraw[draw=gray,fill opacity=0.2,fill=blue!20](.556,-2.305)--(.556,2.005)--(.404,1.949)--(.404,-2.362)--cycle;
\filldraw[draw=gray,fill opacity=0.2,fill=blue!20](-.212,-2.398)--(-.212,1.912)--(-.404,1.949)--(-.404,-2.362)--cycle;
\filldraw[draw=gray,fill opacity=0.2,fill=blue!20](.404,-2.362)--(.404,1.949)--(.212,1.912)--(.212,-2.398)--cycle;
\filldraw[draw=gray,fill opacity=0.2,fill=blue!20](.212,-2.398)--(.212,1.912)--(0,1.9)--(0,-2.41)--cycle;
\filldraw[draw=gray,fill opacity=0.2,fill=blue!20](0,-2.41)--(0,1.9)--(-.212,1.912)--(-.212,-2.398)--cycle;
\filldraw[fill=blue!80,draw=none](-.687,-.431)--(-.692,-.442)--(-.696,-.431)--(-.692,-.42)--cycle;
\draw(-.692,-.442)--(-.696,-.431)--(-.692,-.42);
\filldraw[fill=blue!80,draw=none](.687,-.431)--(.692,-.42)--(.696,-.431)--(.692,-.442)--cycle;
\draw(.692,-.42)--(.696,-.431)--(.692,-.442);
\filldraw[fill=blue!80,draw=none](-.653,-.51)--(-.667,-.5)--(-.692,-.442)--(-.687,-.431)--cycle;
\draw(-.667,-.5)--(-.692,-.442);
\filldraw[fill=blue!80,draw=none](.653,-.51)--(.687,-.431)--(.692,-.442)--(.667,-.5)--cycle;
\draw(.692,-.442)--(.667,-.5);
\filldraw[fill=blue!80,draw=none](.653,-.51)--(.667,-.5)--(.662,-.511)--(.649,-.521)--cycle;
\draw(.667,-.5)--(.662,-.511)--(.649,-.521);
\filldraw[fill=blue!80](.696,-.431)--(.696,.431)--(.662,.351)--(.662,-.511)--cycle;
\filldraw[fill=blue!80,draw=none](-.653,-.51)--(-.649,-.521)--(-.662,-.511)--(-.667,-.5)--cycle;
\draw(-.649,-.521)--(-.662,-.511)--(-.667,-.5);
\filldraw[fill=blue!80](-.662,-.511)--(-.662,.351)--(-.696,.431)--(-.696,-.431)--cycle;
\filldraw[fill=blue!80,draw=none](.556,-.581)--(.653,-.51)--(.649,-.521)--(.577,-.573)--cycle;
\draw(.649,-.521)--(.577,-.573);
\filldraw[fill=blue!80,draw=none](-.556,-.581)--(-.577,-.573)--(-.649,-.521)--(-.653,-.51)--cycle;
\draw(-.577,-.573)--(-.649,-.521);
\filldraw[fill=blue!80,draw=none](-.556,-.581)--(-.542,-.591)--(-.563,-.583)--(-.577,-.573)--cycle;
\draw(-.542,-.591)--(-.563,-.583)--(-.577,-.573);
\filldraw[fill=blue!80](-.563,-.583)--(-.563,.279)--(-.662,.351)--(-.662,-.511)--cycle;
\filldraw[fill=blue!80,draw=none](.556,-.581)--(.577,-.573)--(.563,-.583)--(.542,-.591)--cycle;
\draw(.577,-.573)--(.563,-.583)--(.542,-.591);
\filldraw[fill=blue!80](.662,-.511)--(.662,.351)--(.563,.279)--(.563,-.583)--cycle;
\filldraw[fill=blue!80,draw=none](-.687,.431)--(-.692,.442)--(-.696,.431)--(-.692,.42)--cycle;
\draw(-.692,.442)--(-.696,.431)--(-.692,.42);
\filldraw[fill=blue!80,draw=none](.687,.431)--(.692,.42)--(.696,.431)--(.692,.442)--cycle;
\draw(.692,.42)--(.696,.431)--(.692,.442);
\filldraw[fill=blue!80,draw=none](.653,.352)--(.667,.362)--(.692,.42)--(.687,.431)--cycle;
\draw(.667,.362)--(.692,.42);
\filldraw[fill=blue!80,draw=none](-.653,.352)--(-.687,.431)--(-.692,.42)--(-.667,.362)--cycle;
\draw(-.692,.42)--(-.667,.362);
\filldraw[fill=blue!80,draw=none](.404,-.637)--(.556,-.581)--(.542,-.591)--(.43,-.632)--cycle;
\draw(.542,-.591)--(.43,-.632);
\filldraw[fill=blue!80,draw=none](-.404,-.637)--(-.43,-.632)--(-.542,-.591)--(-.556,-.581)--cycle;
\draw(-.43,-.632)--(-.542,-.591);
\filldraw[fill=blue!80,draw=none](-.404,-.637)--(-.383,-.645)--(-.409,-.64)--(-.43,-.632)--cycle;
\draw(-.383,-.645)--(-.409,-.64)--(-.43,-.632);
\filldraw[fill=blue!80](-.409,-.64)--(-.409,.222)--(-.563,.279)--(-.563,-.583)--cycle;
\filldraw[fill=blue!80,draw=none](.404,-.637)--(.43,-.632)--(.409,-.64)--(.383,-.645)--cycle;
\draw(.43,-.632)--(.409,-.64)--(.383,-.645);
\filldraw[fill=blue!80](.563,-.583)--(.563,.279)--(.409,.222)--(.409,-.64)--cycle;
\filldraw[fill=white,draw=gray](.212,1.912)--(.404,1.949)--(.556,2.005)--(.653,2.076)--(.687,2.155)--(.653,2.234)--(.556,2.305)--(.404,2.362)--(.212,2.398)--(0,2.41)--(-.212,2.398)--(-.404,2.362)--(-.556,2.305)--(-.653,2.234)--(-.687,2.155)--(-.653,2.076)--(-.556,2.005)--(-.404,1.949)--(-.212,1.912)--(0,1.9)--cycle;
\filldraw[fill=blue!80,draw=none](-.212,-.674)--(-.242,-.672)--(-.383,-.645)--(-.404,-.637)--cycle;
\draw(-.242,-.672)--(-.383,-.645);
\filldraw[fill=blue!80,draw=none](.212,-.674)--(.404,-.637)--(.383,-.645)--(.242,-.672)--cycle;
\draw(.383,-.645)--(.242,-.672);
\filldraw[fill=blue!80,draw=none](.653,.352)--(.649,.341)--(.662,.351)--(.667,.362)--cycle;
\draw(.649,.341)--(.662,.351)--(.667,.362);
\filldraw[fill=blue!80,draw=none](-.653,.352)--(-.667,.362)--(-.662,.351)--(-.649,.341)--cycle;
\draw(-.667,.362)--(-.662,.351)--(-.649,.341);
\filldraw[fill=blue!80,draw=none](.186,-.679)--(.212,-.674)--(.242,-.672)--(.215,-.677)--cycle;
\draw(.242,-.672)--(.215,-.677)--(.186,-.679);
\filldraw[fill=blue!80](.409,-.64)--(.409,.222)--(.215,.185)--(.215,-.677)--cycle;
\filldraw[fill=blue!80,draw=none](-.212,-.674)--(-.186,-.679)--(-.215,-.677)--(-.242,-.672)--cycle;
\draw(-.186,-.679)--(-.215,-.677)--(-.242,-.672);
\filldraw[fill=blue!80](-.215,-.677)--(-.215,.185)--(-.409,.222)--(-.409,-.64)--cycle;
\filldraw[fill=blue!80,draw=none](-.556,.281)--(-.653,.352)--(-.649,.341)--(-.577,.289)--cycle;
\draw(-.649,.341)--(-.577,.289);
\filldraw[fill=blue!80,draw=none](.556,.281)--(.577,.289)--(.649,.341)--(.653,.352)--cycle;
\draw(.577,.289)--(.649,.341);
\filldraw[fill=blue!80,draw=none](0,-.686)--(.212,-.674)--(.186,-.679)--(.029,-.688)--cycle;
\draw(.186,-.679)--(.029,-.688);
\filldraw[fill=blue!80,draw=none](0,-.686)--(-.029,-.688)--(-.186,-.679)--(-.212,-.674)--cycle;
\draw(-.029,-.688)--(-.186,-.679);
\filldraw[fill=blue!80,draw=none](0,-.686)--(.029,-.688)--(0,-.69)--(-.029,-.688)--cycle;
\draw(.029,-.688)--(0,-.69)--(-.029,-.688);
\filldraw[fill=blue!80](.215,-.677)--(.215,.185)--(0,.172)--(0,-.69)--cycle;
\filldraw[fill=blue!80](0,-.69)--(0,.172)--(-.215,.185)--(-.215,-.677)--cycle;
\filldraw[fill=blue!80,draw=none](.556,.281)--(.542,.271)--(.563,.279)--(.577,.289)--cycle;
\draw(.542,.271)--(.563,.279)--(.577,.289);
\filldraw[fill=blue!80,draw=none](-.556,.281)--(-.577,.289)--(-.563,.279)--(-.542,.271)--cycle;
\draw(-.577,.289)--(-.563,.279)--(-.542,.271);
\filldraw[fill=blue!80,draw=none](.404,.225)--(.43,.23)--(.542,.271)--(.556,.281)--cycle;
\draw(.43,.23)--(.542,.271);
\filldraw[fill=blue!80,draw=none](-.404,.225)--(-.556,.281)--(-.542,.271)--(-.43,.23)--cycle;
\draw(-.542,.271)--(-.43,.23);
\filldraw[fill=blue!80,draw=none](-.404,.225)--(-.43,.23)--(-.409,.222)--(-.383,.217)--cycle;
\draw(-.43,.23)--(-.409,.222)--(-.383,.217);
\filldraw[fill=blue!80,draw=none](.404,.225)--(.383,.217)--(.409,.222)--(.43,.23)--cycle;
\draw(.383,.217)--(.409,.222)--(.43,.23);
\filldraw[fill=blue!80,draw=none](-.212,.188)--(-.404,.225)--(-.383,.217)--(-.242,.19)--cycle;
\draw(-.383,.217)--(-.242,.19);
\filldraw[fill=blue!80,draw=none](.212,.188)--(.242,.19)--(.383,.217)--(.404,.225)--cycle;
\draw(.242,.19)--(.383,.217);
\filldraw[fill=blue!80,draw=none](-.212,.188)--(-.242,.19)--(-.215,.185)--(-.186,.183)--cycle;
\draw(-.242,.19)--(-.215,.185)--(-.186,.183);
\filldraw[fill=blue!80,draw=none](.212,.188)--(.186,.183)--(.215,.185)--(.242,.19)--cycle;
\draw(.186,.183)--(.215,.185)--(.242,.19);
\filldraw[fill=blue!80,draw=none](0,.176)--(-.212,.188)--(-.186,.183)--(-.029,.174)--cycle;
\draw(-.186,.183)--(-.029,.174);
\filldraw[fill=blue!80,draw=none](0,.176)--(.029,.174)--(.186,.183)--(.212,.188)--cycle;
\draw(.029,.174)--(.186,.183);
\filldraw[fill=blue!80,draw=none](.029,.174)--(0,.176)--(-.029,.174)--(0,.172)--cycle;
\draw(-.029,.174)--(0,.172)--(.029,.174);
\draw[very thick,-latex] (0,.431) -- ++(0,1);
\draw[very thick,yscale=0.8,xscale=1.4,postaction=decorate,decoration={markings,mark=at position .95 with {\arrow{>}}}] (0,.431) ++ (0.3,0) arc (0:180:0.3);
\end{tikzpicture}
}

\clearpage

\task{Электропоезд, составленный из одинаковых вагонов длиной $l$, 
начинает торможение в тот момент, когда первый вагон состава 
заходит в туннель длиной $L$. Двигаясь равнозамедленно, поезд 
останавливается в тот момент, когда его последний вагон выходит 
из туннеля. Известно, что пассажир первого вагона находился в 
туннеле в течение времени $T_1$, а последнего --- $T_N$. Чему равно 
количество вагонов в составе электропоезда?}

\taskpic{$N$ одинаковых небольших шариков подвешены к одной точке на 
невесомых нерастяжимых нитях длиной $L$. В начальный момент все 
маятники находятся в одной плоскости, содержащей вертикаль, 
проходящую через точку подвеса, и отклонены на углы $0 < \varphi_1 < 
\varphi_2 < \ldots < \varphi_n \ll \pi/2$. Начиная с первого, маятники 
последовательно отпускают без начальной скорости в моменты 
времени $\tau_1, \tau_2, \ldots, \tau_n$ соответственно. В какие моменты 
времени последний маятник будет находиться в точке своего 
начального положения? Все удары абсолютно 
упругие.}{
\begin{tikzpicture}
  \draw[interface,thick] (0.2,4) -- ++(3.6,0);
  \draw[very thick,fill=black] (2.5,4) -- ++(-100:3) circle (0.2) node
 [below=5] {1}; 
  \draw[very thick,fill=black] (2.5,4) -- ++(-110:3) circle (0.2)  node
 [below=5] {2};
  \draw[very thick,fill=black] (2.5,4) -- ++(-120:3) circle (0.2)  node
 [below=5] {3};
  \draw[very thick,fill=black] (2.5,4) -- ++(-135:3) circle (0.2)  node
 [below=5] {$n$};
\end{tikzpicture}
}

\vspace{1cm}

\taskpic{Заряженной частице с массой $m$, помещенной в вакууме на 
границе двух областей, в которых созданы однородные магнитные 
поля $B_1$ и $B_2$ ($B_2 > B_1$), сообщают начальную скорость $v_0$, 
направленную перпендикулярно границе раздела. При каких 
значениях заряда частицы ее траектория пройдет через точку M, 
расположенную на расстоянии $L$ от точки 
старта?}{
\begin{tikzpicture}
  \coordinate (a) at (3.5,0.5);
  \coordinate (b) at (0.5,3.5);
  \draw[red] (a) circle (0.2) node[left=5] {$B_2$};
  \draw[red] (a) ++ (-0.1,0.1) -- ++(0.2,-0.2);
  \draw[red] (a) ++ (0.1,0.1) -- ++(-0.2,-0.2);
  \draw[red] (b) circle (0.2) node[right=5] {$B_1$};
  \draw[red] (b) ++ (-0.1,0.1) -- ++(0.2,-0.2);
  \draw[red] (b) ++ (0.1,0.1) -- ++(-0.2,-0.2);
  
  \draw[dashed,thick] (0.2,2) -- ++(3.6,0);
  \draw[very thick,->] (1,2) -- ++(0,-0.75) node[left] {$v_0$};
  \filldraw[black] (3,2) circle (0.05) node[below] {$M$};
  \draw[blue,<->] (1,2.25) -- ++(2,0) node[fill=white,midway,above] {$L$};
\end{tikzpicture}
}

\taskpic{Найдите заряды на конденсаторах в схеме, изображенной на 
рисунке.}{
\begin{circuitikz}
  \draw[thick] (0,0) to[C, l_=$C_1$] (1.8,0) to [C, l_=$C_2$] (3.6,0) --
  ++(0,2.5) to [battery, l_=$\varepsilon_2$] (1.8,2.5);
  \draw[thick] (0,0) -- ++(0,2.5) to [battery,l^=$\varepsilon_1$] (1.8,2.5);
  \draw[thick] (1.8,0) to [C,*-*,l_=$C_3$] (1.8,2.5);
\end{circuitikz}
}

\vspace{1cm}

\task{С помощью кипятильника, рассчитанного на 110~В, можно 
вскипятить воду в чайнике за время $t$. Известно, что превышение 
мощности кипятильника на 20\% приводит к выходу его из строя. Как с 
помощью двух кипятильников на 110~В вскипятить такое же 
количество воды в чайнике, если напряжение в розетке 220~В? Какое 
время потребуется для этого? Потерями тепла пренебречь.}

\taskpic{По поверхности диэлектрической фигуры в форме телефонной 
трубки равномерно распределен заряд $Q > 0$. Фигура помещена в 
электрическое поле напряженностью $E$ так, что она может свободно 
вращаться вокруг точки A. В положении равновесия угол между 
отрезком AB и направлением электрического поля равен $\alpha$. Какую 
работу надо совершить, чтобы медленно повернуть фигуру в 
положение, когда отрезок AB направлен вдоль поля? Длина отрезка AB = 
$L$ , силой тяжести пренебречь.}{
\begin{tikzpicture} 
  \coordinate (a) at ($(2.5,3)!0.2!(2,0.5)$);
  \coordinate (b) at ($(2,0.5)!0.2!(2.5,3)$);
  \draw[shading=ball,rounded corners=3] (2.5,3) to[out=-20,in=0]
  (2,0.5) -- (b) to[out=0,in=-20] (a) -- (2.5,3);
  \draw[blue,<->] (2.4,3) -- (1.9,0.5) node [midway,left] {$L$};
  \draw (2.5,3) node [above] {$A$};
  \draw (2,0.5) node [below] {$B$};
\end{tikzpicture}  
}

\vspace{1cm}

\taskpic{Какую горизонтальную скорость необходимо сообщить очень 
маленькому мячику, лежащему на краю верхней ступеньки лестницы 
для того, чтобы первый отскок мяча произошел от ступеньки с 
номером $N$? Длины и высоты ступенек лестницы соответственно 
равны $b$ и $h$.}{
\begin{tikzpicture}
  \draw[thick] (0.5,3) -- ++(1,0) --++(0,-0.5) -- ++(0.7,0)
  --++(0,-0.5) --++(0.7,0) --++(0,-0.5) --++(0.7,0); 
  \draw[shading=ball] (1.35,3.15) circle (0.15);
  \draw[thick,->] (1.5,3.15) -- ++(0.75,0) node [above] {$v$};
  \draw[blue,->] (0.75,2) node[below] {$h$} to[out=90,in=180]
  (1.4,2.75);
  \draw[blue,->] (1.5,1.5) node [below] {$b$} to [out=90,in=-90] (1.8,2.4);
\end{tikzpicture}
}

\taskpic{Легкая нерастяжимая нить привязана в точке A к потолку, 
затем пропущена сквозь маленькую массивную бусинку, зацеплена 
за два блока B и C, и привязана к бусинке вторым концом. 
Первоначально бусинку удерживают так, что углы, которые нить 
образует с горизонталью, равны $\alpha$, $\beta$ и 90$^\circ$. Затем систему 
отпускают. Найдите вектор ускорения бусинки в начальный момент 
времени. Ускорение свободного падения $g$.}{
\begin{tikzpicture}
  \draw[interface,thick] (0.2,3.5) -- ++(3.6,0);
  \node[draw,circle,thick,label=0:$C$] (r) at (3,3) {};
  \node[draw,circle,thick,label=180:$B$] (l) at (2.25,3) {};
  \draw (r.east) ++ (-0.01,0) -- ++(0,-2) -- (0.5,3.5);
  \draw (r.north)++(0,-0.01) -- ++(-0.75,0);
  \coordinate (c) at ($(r.east) + (-0.01,-2)$);
  \draw (c) -- (tangent cs:node=l,point= {(c)},solution=2);
  \draw[fill=black] (c) circle (0.15);
  \draw (2.1,3.5) -- (l.center) -- (2.4,3.5);
  \draw (2.85,3.5) -- (r.center) -- (3.15,3.5);
  \draw[blue] (1,3.5) arc (0:-atan(2.5/2.65):0.5)
  node[right=9,below=-10] {$\alpha$};
  \draw[blue,double] (2.6,3.2) arc (0:-80:0.5) node[right=4] {$\beta$};
\end{tikzpicture}
}

\vspace{1cm}

\task{Тяжелый теплоизолированный контейнер массой $M = 10\mbox{ кг}$, 
содержащий $m = 1\mbox{ кг}$ газа, отпустили без начальной скорости с 
высоты $H = 15\mbox{ м}$. На какую высоту подскочит контейнер, если его 
соударение с землей абсолютно упругое? Сопротивлением воздуха 
пренебречь; считать, что колебания в газе быстро затухают.}

\task{Вертикально в землю вкопан длинный стержень, по которому 
могут без трения двигаться две маленькие бусинки. Бусинки упруго 
соударяются друг с другом, а нижняя упруго соударяется с землей. 
Верхняя, которая в $n = 10^4$ раз тяжелее нижней, практически 
неподвижно зависла на высоте $H = 1\mbox{ м}$ над Землей. Определите 
скорость нижней бусинки.}

\vspace{1cm}

\taskpic{Тележка массой $M$, двигающаяся со скоростью $V$ прямолинейно, 
наталкивается на легкую пружину длиной $L$, прикрепленную к стене. 
На тележке закреплен хрупкий предмет массы $m$, который 
разбивается, если его перемещать с ускорением больше чем $a_0$. 
Какой должна быть жесткость пружины, чтобы в процессе 
столкновения хрупкий предмет не разбился? Трением тележки о пол 
пренебречь. Пружину считать идеальной при любой длине от 0 до 
$L$.}{
\begin{tikzpicture}
  \draw[thick,interface] (3.8,3) -- ++ (0,-3) -- ++ (-3.6,0);
  \draw[thick] (1,0.25) circle (0.25);
  \draw[thick] (2,0.25) circle (0.25);
  \draw[thick] (0.5,0.5) node[above=7] {$M$} rectangle ++(2,0.25);
  \draw[pattern=dots] (1.25,0.75) rectangle ++(0.5,1) node [above left]
  {$m$};
  \draw[spring] (3.8,1.25/2) -- ++(-0.8,0) node [midway,above=5] {$L$};
  \draw[fill=black] (3,1.75/2) rectangle ++(-0.1,-0.5);
\end{tikzpicture}
}


\taskpic{Зависимость тока от напряжения на элементе X приведена на 
графике. Постройте график зависимости тока от напряжения для 
схемы, изображенной на рисунке. Схема состоит из резисторов 
(номиналы указаны) и элементов 
X.}{
\begin{circuitikz}
  \draw[thick] (0.2,2) -- ++(0.3,0) -- ++(0,1) to [generic,l=$R$] (2,3) to
  [Do,l=$X$] (3.5,3) -- (3.5,2) -- (3.8,2);
  \draw[thick] (0.5,2) -- ++(0,-1) to [Do,l=$X$] (2,1) to [generic,l=$R$]
  (3.5,1) -- ++(0,1);     
\end{circuitikz}
}
\begin{figure}[h]
  \centering
  \begin{tikzpicture}
    \draw[thick,->] (0.4,0.5) -- ++(2.5,0) node[above] {$U$};
    \draw[thick,->] (0.5,0.4) -- ++(0,2.5) node[right] {$I$};
    \draw[red,very thick] (0.5,0.5) -- ++(1.5,0) -- ++(0,2);
    \draw (2,0) node {$U_0$};
  \end{tikzpicture}
\end{figure}

\vspace{1cm}

\taskpic{К сосуду подключен нагреватель постоянной мощности, и в 
него налито некоторое количество жидкости. Дан график 
зависимости температуры жидкости от времени. Жидкость в сосуде 
хорошо перемешивается, поэтому можно считать температуру 
одинаковой по всему объему. Опыт повторяют с теми же начальными 
условиями, однако теперь в момент времени $t_1 = 5\mbox{ мин}$ массу 
жидкости увеличивают вдвое, не меняя ее температуру. Найдите при 
этом температуру жидкости в момент времени $t_2 = 10\mbox{ мин}$. 
Считайте, что мощность тепловых потерь не зависит от объема 
жидкости.}{
\begin{tikzpicture}[domain=0:3]
  \draw[help lines,xstep=0.75,ystep=0.5] (0,0) grid (3.1,2.8);
  \draw[->] (0,0) --++(3.2,0);
  \draw (1.5,-0.7) node {\tiny{$t$, мин}};
  \draw[->] (0,0) --++(0,3) node[right] {\tiny{$T, ^\circ$C}};
  \draw[red,thick] plot(\x,{2-exp(-1.5*\x)});
  \foreach \x in {2.5,5,7.5,10}
  \draw (\x*3/10,0.05) -- ++(0,-0.1) node[anchor=north] {\tiny{$\x$}};
  \foreach \y in {0,20,40,60,80,100}
  \draw (0.05,\y/40) -- ++(-0.1,0) node[anchor=east] {\tiny{$\y$}};
\end{tikzpicture}
}

\task{Воздушный шарик представляет собой тонкую резиновую 
оболочку, имеющую в нерастянутом состоянии радиус $r_0$ и толщину 
$d_0$. Материал шарика обладает модулем Юнга $E$ и при растяжении 
подчиняется следующему закону: произведение относительного 
удлинения на модуль Юнга равно механическому напряжению. Кроме 
того, при растяжении объем материала не изменяется. Шарик 
помещают в вакуум и начинают надувать газом с молярной массой $\mu$ 
и температурой $T$. Определите зависимость радиуса шарика и 
давления в нем от массы $m$ закачанного газа. Постройте графики 
этих зависимостей. (Напоминаем, что механическое напряжение $\sigma$ 
равно отношению растягивающей силы к площади поперечного 
сечения образца).}

\vspace{1cm}

\taskpic{С гладкого цилиндра радиуса $R$ соскальзывает тонкая 
однородная веревка длины $\pi R/2$, лежащая в вертикальной 
плоскости. Найдите силу натяжения веревки в точке A ($0 < \alpha < \pi/2$). 
Масса веревки $m$, ускорение свободного падения 
$g$.}{
\begin{tikzpicture}
  \draw (2,2) circle (1.5);
  \draw[line width=3,black] (3.55,2) arc (0:90:1.55);
  \draw[thick,blue,dashed] (3.5,2) -- ++(-1.5,0) -- ++(40:1.5) node [above
  right,black] {$A$}; 
  \draw[blue] (2.5,2) arc (0:40:0.5);
  \draw[blue] (2.7,2.2) node {$\alpha$};
\end{tikzpicture}
}

\taskpic{В длинном цилиндрическом сосуде находится $\nu = 1\mbox{
моль}$ гелия, заключенный между теплоизолированными боковой стенкой и
поршнем, соединенных друг с другом резинкой с нулевой начальной
длинной. Коэффициент жесткости резинки равен $k = 3{,}46\mbox{
Н/м}$. Весь цилиндр помещен в очень горячую однородную среду. Поршень
удерживают на расстоянии $l = 1\mbox{ м}$ от левого торца, при этом
температура газа возрастает на $t = 1\mbox{ К}$ за каждые $\tau =
20\mbox{ с}$.  Поршень отпускают; в начальный момент силы, действующие
на поршень, скомпенсированы. Постройте график зависимости длины
резинки $L$ от времени. Трения между поршнем и цилиндром нет. $R =
8{,}31\mbox{ Дж/(моль}\cdot\mbox{К)}$.}{
\begin{tikzpicture}[domain=0:3]
  \draw[pattern=north east lines] (3.8,2.5) --++(0,0.5) --++(-3.8,0)
  -- ++(0,-2) -- ++(3.8,0) -- ++(0,0.5) -- ++(-3.3,0) --++(0,1)
  --++(3.3,0) -- cycle;
  \draw (3.8,2.5) -- (0.5,2.5) -- (0.5,1.5) -- (3.8,1.5);
  \filldraw[gray] (0.5,2.5) rectangle ++(0.2,-1);
  \draw[pattern=dots,pattern color=green] (0.7,2.5) rectangle
  ++(2,-1);
  \draw[very thick] (0.7,2) -- ++(2,0);
  \filldraw[black] (2.7,2.5) rectangle ++(0.2,-1);
  \draw[blue,thick,->] (2,3.5) node[right] {$L$} to[out=180,in=90] (1.5,2.1);
\end{tikzpicture}
}

\vspace{1cm}

\taskpic{Однородный шар массой $M$, равномерно заряженный по объему 
электрическим зарядом $Q$, закреплен на невесомом нерастяжимом 
канате и вращается в плоскости, перпендикулярной линиям 
магнитного поля $B$ с постоянной угловой скоростью $\omega$. Известно, 
что при увеличении длины каната в $n$ = 3~раза сила его натяжения 
увеличивается в $k$ = 2~раза. Чему равна величина вектора магнитной 
индукции, если известно, что после выключения магнитного поля 
сила натяжения нити стала такой же, как до увеличения длины нити? 
Сила тяжести отсутствует.}{
\begin{tikzpicture}
  \draw[dashed,blue] (2,2) circle (1.5);
  \draw[fill=brown] (2,2) circle (0.05);
  \draw[very thick] (2,2) -- ++(225:1.5);
  \shadedraw[shading=ball] ($(2,2)+(225:1.5)$) circle (0.2) node[right=5]
  {$M,Q$};
  \draw[red] (3,3.5) circle (0.15) node[right] {$B$};
  \draw[red] ($(3,3.5) +(0.05,0.05)$) -- ++(-0.1,-0.1);
  \draw[red] ($(3,3.5) +(-0.05,0.05)$) -- ++(0.1,-0.1);
\end{tikzpicture}
}

\task{Тонкая массивная шайба надета на длинный стержень радиуса $R$. 
Когда шайбу закрутили вокруг стержня с угловой скоростью $\omega$, 
оказалось, что она останавливается через время $t_0$. В другой раз 
шайбу закрутили с той же угловой скоростью и одновременно 
придали ей скорость $v_0$ вдоль стержня. Какой путь пройдет по 
стержню шайба до остановки? Зазора между шайбой и стержнем нет.}

\end{document}
