\documentclass[11pt]{article}
% math symbols
\usepackage{amssymb,amsmath}
% for different compilers
\usepackage{ifpdf}
% geometry of page
\usepackage[margin=2.1cm]{geometry}
% float pictures
\usepackage{wrapfig}

% if pdflatex, then
\ifpdf
 \usepackage[english,russian]{babel}
 \usepackage[utf8]{inputenc}
 \usepackage[unicode]{hyperref}
 \usepackage[pdftex]{graphicx}
 \usepackage{cmlgc}
% if xelatex, then
\else
% math fonts
 \usepackage{fouriernc}
% xelatex specific packages
 \usepackage[xetex]{hyperref}
 \usepackage{xunicode}	% some extra unicode support
 \usepackage{xltxtra}	% \XeLaTeX macro
 \defaultfontfeatures{Mapping=tex-text}
 \usepackage{polyglossia}	% instead of babel in xelatex
 \setdefaultlanguage{russian}
% fonts
 \setromanfont{Charis SIL}
 \setsansfont{OfficinaSansC} 
 \setmonofont{Consolas}
\fi

% several pictures in one figure
\usepackage{subfig}
% calc in TeX expressions
\usepackage{calc}
% nice pictures and plots
\usepackage{pgfplots,tikz,circuitikz}
% different libraries for pictures
\usetikzlibrary{%
  arrows,%
  calc,%
  patterns,%
  decorations.pathreplacing,%
  decorations.pathmorphing,%
  decorations.markings%
}

% colors of the hyperlinks
\hypersetup{colorlinks,%
  citecolor=blue,%
  urlcolor=blue,%
  linkcolor=red
}

\tolerance=1000
\emergencystretch=0.74cm

\newcommand{\nn}{\nonumber}
\newcommand{\pt}{\partial}
\newcommand{\eps}{\epsilon}
\newcommand{\vareps}{\varepsilon}
\newcommand{\const}{\mathrm{const}}
\newcommand{\com}[1]{{\Large{\texttt{{\color{red}(#1)}}}}}

% счётчик задач
\newcounter{notask}
\setcounter{notask}{1}

% условие без картинки
\newcommand{\task}[1]{
	\hrule
	\hbox to \textwidth {%
    	\vrule
	    \parbox[t]{0.04\textwidth}{\smallskip \centering \arabic{notask}}%
    	\vrule%
	    \hfill%
    	\parbox[t]{0.93\textwidth}{\smallskip #1 \smallskip}\hfill%
	    \vrule
	}
	\hrule
    \addtocounter{notask}{1}
    \pagebreak[2]
}

\newlength{\h}
\newsavebox{\taskbox}
\newlength{\x}
\newsavebox{\pictbox}

% условие с картинкой (картинка выравнивается по центру)
\newcommand{\taskpic}[2]{
	\savebox{\taskbox}{\parbox[t]{0.93\textwidth-4.3cm}{\smallskip #1 \smallskip}}
	\savebox{\pictbox}{\parbox[t]{4cm}{\smallskip \centering
    	\vspace{0pt} #2 \smallskip}}
	\h=\ht\taskbox
	\advance\h\dp\taskbox
	\x=\ht\pictbox
	\advance\x\dp\pictbox
	\hrule
	\hbox to \textwidth {%
		\vrule\parbox[t][\maxof{\h}{\x}][t]{0.04\textwidth}{ \smallskip
    		\centering \arabic{notask} }\vrule%
	    \hfill\parbox[t][\maxof{\h}{\x}][t]{0.93\textwidth-4.3cm}{\smallskip #1
    	    \smallskip}\hfill\vrule%
	    \hfill\parbox[t][\maxof{\h}{\x}][c]{4cm}{\hfil #2 \hfil}\hfill\vrule
	}
	\hrule
	\addtocounter{notask}{1}
	\pagebreak[2]
}

\tikzset{>=latex, interface/.style = {postaction = {draw, decorate,
	    decoration = {border, angle = 45, amplitude = 0.2cm, segment length=1.4323mm}}},%
		spring/.style={decorate,decoration={snake,amplitude=1mm,
        segment length=2mm},thick}}


\begin{document}

\thispagestyle{empty}
\parindent=5mm
\righthyphenmin=2
\begin{center}

\phantom{Бухалкин апанас}

\vfill
\LARGE{\textsc{XVII Летняя Физическая Школа}}\\
\Large{\textsc{15 июля -- 4 августа 2011}}\\[1cm]
\Large{\textit{Сборник материалов}}\\[2cm]
\includegraphics[width=6cm]{logo.pdf}
\vfill

\small{\textsf{Санкт-Петербург}\\
\textsf{2011}}
\end{center}

\clearpage

\section{Участники ЛФШ.}
\label{sec:participants}

\subsection{8 и 9 класс.}
\label{sec:8and9}

\begin{table}[ht]
  \begin{minipage}[t]{0.45\linewidth}\centering
    \begin{center}
      \textit{8 класс}
    \end{center}
    \begin{tabular}[h]{|c|c|c|c|}
      \hline
      \textbf{№} & \textbf{Фамилия, имя} & \textbf{Шк.} & \textbf{Д/б}\\
      \hline
      1 & Артамонов Николай & 239 & 47\\ \hline
      2 & Багиров Фарид & 533 & Д1 \\ \hline
      3 & Баринов Фёдор & 239 & Д3 \\ \hline
      4 & Богомолов Егор &  92 & ПО \\ \hline
      5 & Бомов Фёдор & 292 & 56 \\ \hline
      6 & Галкин Георгий & --- & --- \\ \hline
      7 & Гера Станислава & 534 & Д3 \\ \hline
      8 & Грибакин Борис & 74 & 50 \\ \hline
      9 & Громов Даниил & Ун. & 24 \\ \hline
      10 & Гуменюк Виталий & 92 & 66 \\ \hline
      11 & Жибарев Георгий & 533 & Д3 \\ \hline
      12 & Йона Андрей & 152 & --- \\ \hline
      13 & Кокурушников Тимофей & 239 & Д2 \\ \hline
      14 & Коробов Артём & 2 & 48 \\ \hline
      15 & Кутимский Максим & 239 & --- \\ \hline
      16 & Лебедев Александр & --- & 14 \\ \hline
      17 & Мастеров Роман & 366 & 58 \\ \hline
      18 & Морозов Дмитрий & 239 & Д2 \\ \hline
      19 & Олейник Дарья & 239 & 53 \\ \hline
      20 & Охотников Артём & 239 & Д2 \\ \hline
      21 & Петров Степан & 30 & Д2 \\ \hline
      22 & Приходько Алексей & 366 & ПО \\ \hline
      23 & Родионова Анна & 239 & 54 \\ \hline
      24 & Рутковский Илья & 78 & 46 \\ \hline
      25 & Сокольский Станислав & 366 & 33 \\ \hline
      26 & Староказников Александр & 344 & ПО \\ \hline
      27 & Трофимов Даниил & 101 & Д3 \\ \hline
      28 & Уланова Арина & 533 & Д2 \\ \hline
      29 & Усачёва Мария & 295 & 23 \\ \hline
      30 & Ходунов Павел & 239 & Д1 \\ \hline
      31 & Шубин Григорий & 550 & 42 \\ \hline
    \end{tabular}
  \end{minipage}
  \hfill
  \begin{minipage}[t]{0.45\linewidth}\centering
    \begin{center}
      \textit{9 класс}
    \end{center}
    \begin{tabular}[h]{|c|c|c|c|}
      \hline
      \textbf{№} & \textbf{Фамилия, имя} & \textbf{Шк.} & \textbf{Д/б}\\
      \hline
      1 & Агафонов Игорь & 533 & ПО  \\ \hline
      2 & Андреев Константин & ФТШ & Д2  \\ \hline
      3 & Беляков Михаил & 239 & Д1  \\ \hline
      4 & Блехштейн Максим & 239 & Д3  \\ \hline
      5 & Буренев Иван & ФТШ & Д3  \\ \hline
      6 & Вахренев Роман & 239 & 44  \\ \hline
      7 & Глезеров Евгений & 73 & ---  \\ \hline
      8 & Гордеева Людмила & ФТШ & 16  \\ \hline
      9 & Гуцол Ксения & ФТШ & ПО  \\ \hline
      10 & Затылкин Павел & 239 & 47  \\ \hline
      11 & Затылкин Кирилл & 239 & 40  \\ \hline
      12 & Иванов Владислав & 177 & Д3  \\ \hline
      13 & Киселев Егор & 239 & Д3  \\ \hline
      14 & Коско Софья & 239 & Л  \\ \hline
      15 & Кравченко Дмитрий & --- & 40  \\ \hline
      16 & Куксенок Даниил & 533 & ПО  \\ \hline
      17 & Лихачев Иван & 30 & 44  \\ \hline
      18 & Луцкий Георгий & ФТШ & ПО  \\ \hline
      19 & Малышева Александра & ФТШ & 39  \\ \hline
      20 & Муретова Мария & 239 & Д1  \\ \hline
      21 & Никоненко Михаил & 393 & ---  \\ \hline
      22 & Осипов Игорь & 571 & ---  \\ \hline
      23 & Портянкин Егор & ФТШ & 65  \\ \hline
      24 & Семенов Александр & ФТШ & 31  \\ \hline
      25 & Сычёв Станислав & ФТШ & Д1  \\ \hline
      26 & Хвещук Анастасия & 470 & ---  \\ \hline
      27 & Цейтина Елена & ФТШ & 41  \\ \hline
    \end{tabular}
  \end{minipage}
\end{table}

В \textbf{8 классе} работали: Н.В. Тараканов, М.В. Евтихиев (студент 3-го
курса ФТФ СПбГПУ), А.В. Лиознова (студентка 3-го курса ФТФ СПбГПУ),
И. Авдеев, А. Лиознов (студенты 1-го курса ФТФ СПбГПУ), Д. Максимова
(студентка 1-го курса ФТФ СПбГПУ), С. Богданов (студент 1-го курса ФФ
МГУ).

В \textbf{9 классе} работали: О.В. Шустова (студентка 6-го курса ФТФ СПбГПУ),
Ф. Затылкин (студент 2-го курса ФМФ СПбГПУ), А. Коротченков (студент
2-го курса ФТФ СПбГПУ), Ф. Петухов (студент 1-го курса ФТФ СПбГПУ). 

\clearpage

\subsection{10 и 11 класс.}
\label{sec:10and11}

\begin{table}[ht]
  \begin{minipage}[t]{0.45\linewidth}\centering
    \begin{center}
      \textit{10 класс}
    \end{center}
    \begin{tabular}[h]{|c|c|c|c|}
      \hline
      \textbf{№} & \textbf{Фамилия, имя} & \textbf{Шк.} & \textbf{Д/б}\\
      \hline
      1 & Бальков Андрей & 261 & 37\\ \hline
      2 & Вишняк Сергей & ФТШ & 41 \\ \hline
      3 & Водопьян Даниил & 239 & Р \\ \hline
      4 & Грудкин Антон & 239 & Д1 \\ \hline
      5 & Егоров Антон & ФТШ & Р \\ \hline
      6 & Иващенко Дмитрий & 239 & 32 \\ \hline
      7 & Капралов Николай & ФТШ & 46 \\ \hline
      8 & Конаныхин Роман & ФТШ & Д1 \\ \hline
      9 & Крюков Михаил & ФТШ & 31 \\ \hline
      10 & Лашкевич Злата & ФТШ & 24 \\ \hline
      11 & Люлина Анастасия & ФТШ & 27 \\ \hline
      12 & Максакова Мария & ФТШ & 39 \\ \hline
      13 & Мосягин Иван & 214 & 42 \\ \hline
      14 & Погодаев Илья & 610 & ПО \\ \hline
      15 & Рау Владислава & 239 & 46 \\ \hline
      16 & Серов Юрий & ФТШ & Р \\ \hline
      17 & Томп Дмитрий & ФТШ & Д3 \\ \hline
      18 & Ярковой Алексей & 557 & 24 \\ \hline
    \end{tabular}
  \end{minipage}
  \hfill
  \begin{minipage}[t]{0.45\linewidth}\centering
    \begin{center}
      \textit{11 класс}
    \end{center}
    \begin{tabular}[h]{|c|c|c|c|}
      \hline
      \textbf{№} & \textbf{Фамилия, имя} & \textbf{Шк.} & \textbf{Д/б}\\
      \hline
      1 & Балашов Александр & 30 & ---  \\ \hline
      2 & Борздун Наталья & 239 & ---  \\ \hline
      3 & Грачёв Дмитрий & М & ---  \\ \hline
      4 & Жаровов Дмитрий & 239 & ---  \\ \hline
      5 & Косицын Александр & 239 & ---  \\ \hline
      6 & Максимишин Дмитрий & 239 & ---  \\ \hline
      7 & Маслов Артём & ФТШ & Р  \\ \hline
      8 & Матюшин Георгий & ФТШ & Д1  \\ \hline
      9 & Михайлов Кирилл & 239 & ---  \\ \hline
      10 & Никитин Денис & 239 & Р  \\ \hline
      11 & Свирина Анна & 239 & ---  \\ \hline
      12 & Смирнов Иван & 214 & Д2  \\ \hline
      13 & Терехов Антон & 239 & Р  \\ \hline
      14 & Толстопятов Всеволод & 239 & ---  \\ \hline
      15 & Трофимов Павел & ФТШ & ---  \\ \hline
      16 & Чурилова Мария & ФТШ & ---  \\ \hline
      17 & Шалымов Роман & 239 & Д3  \\ \hline
    \end{tabular}
  \end{minipage}
\end{table}

В \textbf{10 классе} работали: И.А. Барыгин (к.ф.--м.н., преподаватель
ФТШ), С. Атамась, В. Коваленко (студенты 3-го курса ФТФ СПбГПУ).

В \textbf{11 классе} работали: И.Е. Шендерович, Д.О. Соколов
(аспиранты ПОМИ РАН), Д.С. Смирнов (студент 4-го курса ФТФ СПбГПУ). 

\section{Ежедневные занятия.}
\label{sec:daily}

\subsection{8 класс.}
\label{sec:daily8}

\subsubsection{Теория.}
\label{sec:th8}

\textit{Преподаватель: Н.В. Тараканов.}

\subsubsection{Эксперимент.}
\label{sec:exp8}

\textit{Преподаватели: А. Лиознова, Д. Максимова, И. Авдеев.}

\subsection{9 класс.}
\label{sec:daily9}

\subsubsection{Теория.}
\label{sec:th9}

\textit{Преподаватель: О.В. Шустова.}

\subsubsection{Эксперимент.}
\label{sec:exp9}

\textit{Преподаватели: Ф. Затылкин, Ф. Петухов.}

\subsection{Смешанная группа 8--9 классов.}
\label{sec:daily89}

\subsubsection{Теория.}
\label{sec:th89}

\textit{Преподаватель: Н.В. Тараканов.}

\subsubsection{Эксперимент.}
\label{sec:exp89}

\textit{Преподаватели: С. Богданов, А. Лиознов.}

\subsection{10 класс.}
\label{sec:daily10}

\subsubsection{Теория.}
\label{sec:th10}

\textit{Преподаватель: С. Атамась.}

\subsubsection{Эксперимент.}
\label{sec:exp10}

\textit{Преподаватель: И.А. Барыгин.}\\

\begin{enumerate}
\item Измерить отношение масс двух грузов.\\
  \textit{Оборудование:} два груза, нитки, миллиметровка.
\item ВАХ нелинейного элемента.\\
  \textit{Оборудование:} источник постоянного тока, провода, реостат,
  мультиметр, лампочка.
\item Измерение скорости вытекания воды из крана.\\
  \textit{Оборудование:} линейка.
\item Измерение отношения длин ниток Y-образного маятника.\\
  \textit{Оборудование:} нитки, грузики.
\item Измерить жесткость каучукового шарика.\\
  \textit{Оборудование:} каучуковый шарик, вода, линейка, гуашь.
\item Измерить коэффициент трения линейки по столу.\\
  \textit{Оборудование:} две деревянные линейки.
\item Измерить зависимость мощности теплоотдачи от разницы температур.\\
  \textit{Оборудование:} горячая вода, мерный стакан или линейка,
  термометр для воды, часы, сосуд.
\end{enumerate}

\subsection{11 класс.}
\label{sec:daily11}

\subsubsection{Теория.}
\label{sec:th11}

\textit{Преподаватели: И.Е. Шендерович, Д.О. Соколов.}

\begin{enumerate}
\item Введение. Требующие объяснения экспериментальные факты.
\item Векторный анализ. Циркуляция и поток. Градиент, дивергенция и
  ротор. Теорема Стокса и теорема Гаусса--Остроградского.
\item Электростатика. Теорема Гаусса. Потенциал электростатического
  поля.
\item Переход к движущимся зарядам. Уравнение неразрывности
  электрического заряда, его физический смысл.
\item Уравнение Максвелла для ротора магнитного
  поля. Магнитостатика. Закон Био~--~Савара~--~Лапласа, закон
  Ампера. Примеры: поле кругового тока, поле соленоида, поле длинного
  провода.
\item Ток смещения, его физический смысл. Задача о разрядке
  конденсатора.
\item Электромагнитные волны. Закон индукции Фарадея как необходимое
  условие для их существования. Сила Лоренца. Примеры.
\item Физика индукции. Примеры возникновения ЭДС индукции.
\item Движущееся электромагнитное поле. Скорость волны и скорость
  света.
\item Пример полного решения уравнений Максвелла: переменное поле в
  цилиндрическом конденсаторе. Функция Бесселя.
\item Применение уравнений Максвелла к электрическим цепям. Законы
  Кирхгофа. Примеры расчёта схем. 
\end{enumerate}

\subsubsection{Эксперимент.}
\label{sec:exp11}

\textit{Преподаватель: Д.С. Смирнов.}\\

Идеи большинства экспериментов взяты из книги С.Д. Варламова,
А.Р. Зильбермана, В.И. Зинковского <<Экспериментальные задачи на
уроках физики и физических олимпиадах>>.

\begin{enumerate}
\item Найти отношение массы монеты к массе листа миллиметровки.\\
  \textit{Оборудование:} монета, миллиметровка.
\item Найти закон распределения времени, отмеряемого внутренними
  часами. Экспериментатор должен сто раз отмерить по внутренним часам
  10 секунд и сравнить результат с истинным.
\item Определить плотность и массу куска пластилина.\\
  \textit{Оборудование:} мерный стакан, миллиметровка.
\item Измерить расстояние между бороздками CD.\\
  \textit{Оборудование:} лазерная указка известной длины волны,
  линейка.
\item Измерить и объяснить ВАХ лампочки накаливания.\\
  \textit{Оборудование:} батарейка 9В, реостат 10КОм, провода, мультиметр. 
\item Определить схему и номиналы элементов в чёрном ящике.\\
  \textit{Оборудование:} мультиметр. 
\item Определить теплоту плавления припоя.\\
  \textit{Оборудование:} свечка, мерный стакан, термопара, вода. 
\item Определить отношение частот всех мод двойного
  маятника.\\
  \textit{Оборудование:} пластилин, нитки, секундомер.
\end{enumerate}


\section{Материалы физических боёв.}
\label{sec:battles}

\subsection{Физбой 9 класса.}
\label{sec:battle9}

\subsection{Полуфинал.}
\label{sec:battle9s}

\subsection{Финал.}
\label{sec:battle9f}

\clearpage
\subsection{Физбой 10 класса.}
\label{sec:battle10}

\task{С какой силой отталкиваются грани равномерно заряженного по
  поверхности правильного тетраэдра? Длина ребра тетраэдра $a$, заряд
  каждой грани $q$.}

\taskpic{Участок цепи постоянного тока состоит из трех одинаковых
  вольтметров и двух одинаковых амперметров. Показания вольтметров
  V$_1$ и V$_2$ равны $U_1 = 6$~B, $U_2 = 4$~B. Как вы полагаете, что
  показывает третий вольтметр?}{
  \begin{tikzpicture}
    \draw[thick] (0,0.5) -- (0.5,0.5);
    \draw[thick] (0.5,0.5) -- (0.5,2) -- (1.45,2) (2.05,2) -- (2.5,2) --
    (2.5,1) -- (2.7,1);
    \draw[thick] (1.75,2) circle (0.3cm) node[blue] {$V_1$};
    \draw[thick] (3,1) circle (0.3cm) node[blue] {$A_2$};
    \draw[thick] (3.3,1) -- (3.5,1) -- (3.5,0.5) -- (4,0.5);
    \draw[thick] (0.5,1) -- (0.7,1);
    \draw[thick] (1,1) circle (0.3cm) node[blue] {$A_1$};
    \draw[thick] (2.6,1) -- (2.3,1);
    \draw[thick] (2,1) circle (0.3cm) node[blue] {$V_2$};
    \draw[thick] (1.3,1) -- (1.7,1);
    \draw[thick] (0.5,0.5) -- (0.5,0) -- (1.5,0) (2.1,0) -- (3.5,0) --
    (3.5,1);
    \draw[thick] (1.8,0) circle (0.3cm) node[blue] {$V_3$};
  \end{tikzpicture}
}

\task{Ракета пришельцев стартует с поверхности Земли и практически
  мгновенно набирает постоянную вертикальную скорость $v =
  800$~м/с. Злодеи обстреливают ракету из пушки, которая находится в
  $l=40$~км от стартовой площадки. Выстрел производится в момент
  старта. Могут ли злодеи попасть в ракету? Начальная скорость снаряда
  равна $u = 1000$~м/с. Сопротивлением воздуха пренебречь.}

\task{Узкий пучок протонов налетает на шар радиуса $a$. Прицельное
  расстояние (расстояние от центра шара до прямой, на которой лежит
  начальная скорость протонов) $b$, начальная энергия протонов
  $E$. Найти установившийся заряд шара. Заряд протона $e$, масса $m$.}

\taskpic{Кубик массы $M$ стоит на горизонтальной поверхности. Его
  касается кубик массы $m$, висящий на невесомой нерастяжимой
  нити. Нить составляет угол $\alpha$ с вертикалью. В начальный момент
  кубики неподвижны. Определите ускорения кубиков в начальный
  момент. Трением пренебречь. Считайте, что кубики не поворачиваются
  вокруг своей оси.}{
  \begin{tikzpicture}
    \draw[thick,interface] (4,0) -- (0,0);
    \draw[thick] (1.5,0) rectangle (3.5,2) node[midway,blue] {$M$};
    \draw[thick] (0.75,1.25) rectangle (1.5,2) node[midway,blue]
    {$m$};
    \draw (1.12,2) -- (2,4);
    \draw[fill=black] (2,4) circle (0.04cm);
    \draw[dashed] (2,4) -- (2,3);
    \draw[blue] (2,3.25) arc (270:246:0.75);
    \draw[blue] (1.8,3.03) node {$\alpha$};
  \end{tikzpicture}
}


\task{На достаточно удаленные предметы смотрят через собирающую линзу
  с фокусным расстоянием $F=9$~см, располагая глаз на расстоянии
  $a=36$~см от линзы. Оцените минимальный размер экрана, который нужно
  расположить за линзой так, чтобы он перекрыл все поле
  изображения. Считайте, что радиус зрачка равен $r=1{,}5$~мм.}


\clearpage
\subsection{Физбой 11 класса.}
\label{sec:battle11}

\setcounter{notask}{1}
\taskpic{ Два одинаковых кубика с ребром $H$ и массой $2.5m$ каждый стоят
  почти соприкасаясь гранями на гладкой горизонтальной
  поверхности. Сверху на них аккуратно кладут шар массы $m$ и радиуса
  $R$, и он начинает смещаться вертикально вниз, раздвигая кубики в
  стороны. Найти скорость шара непосредственно перед ударом о
  горизонтальную поверхность. Начальная скорость шара пренебрежимо
  мала.  }{
  \begin{tikzpicture}
    \draw[very thick,interface] (4,0) -- (0,0);
    \draw[very thick] (0.5,0) rectangle (1.5,1);
    \draw[very thick] (2.5,0) rectangle (3.5,1);
    \draw[very thick] (2,1.5) circle (0.7cm);
    \draw[blue,<->] (0.75,0) -- (0.75,1) node[midway,right] {$H$};
    \draw[blue,->] (2,1.5) node[above=0.1cm] {$R$} -- ++(40:0.7cm) ; 
  \end{tikzpicture}
}


\taskpic{ За линзой на расстоянии $\ell = 4$ см (больше фокусного)
  расположено перпендикулярно главной оптической оси плоское
  зеркало. Перед линзой, также перпендикулярно главной оптической оси,
  расположен лист клетчатой бумаги. На этом листе получают изображение
  его клеток при двух положениях листа относительно линзы. Эти
  положения отличаются на $L = 9$ см. Определить фокусное расстояние
  линзы.  }{
  \begin{tikzpicture}
    \draw[very thick] (0,0) -- (3.5,0);
    \draw[very thick] (1,1.1) -- (1,-1.1);
    \draw[very thick,<->] (2.5,1.1) -- (2.5,-1.1);
    \draw[very thick,interface] (3.5,1.1) -- (3.5,-1.1);
    \draw[blue,<->] (2.5,-1.2) -- (3.5,-1.2) node[midway,below] {$\ell$};
  \end{tikzpicture}
}

\task{ Вагон массой $M$ и длиной $L$ может без трения двигаться по
  рельсам. Он заполнен газом и разделен пополам подвижной невесомой
  вертикальной перегородкой. Вначале температура газа равна $T$. В
  правой половине включают нагреватель и доводят температуру газа до
  $2T$, в левой части температура остается прежней. Найти перемещение
  вагона, если масса всего газа равна $m$.  }

\task{ Звезда массы $M$ и радиуса $r$ образовалась из однородного
  облака газа с молярной массой $\mu$ радиуса $R$, исходно имевшего
  температуру $T$. Считая звезду также однородной, определите
  среднеквадратичное значение ее угловой скорости.  }

\task{ Радиусы кривизны двух одинаковых, слипшихся друг с другом,
  мыльных пузырей равны $R$. После того, как перегородка лопнула,
  образовался один пузырь радиусом $R_1$. Коэффициент поверхностного
  натяжения мыльного раствора $\sigma$. Найти атмосферное давление.  }

\taskpic{ Имеется равномерно заряженная диэлектрическая
  сфера. Известно, что, если ее разрезать пополам, то <<половинки>>
  будут расталкиваться с силой $F_1$. Если разрезать пополам одну из
  половинок (вдали от второй), то получившиеся <<четвертинки>> будут
  расталкиваться с силой $F_2$. И, наконец, если разрезать пополам
  одну из <<четвертинок>> (вдали от оставшихся частей сферы) на
  <<восьмушки>>, то они будут расталкиваться с силой $F_3$. Найти
  силу, с которой будут расталкиваться <<восьмушки>>, если их
  поместить так, как показано на рисунке.  }{\begin{tikzpicture}

    \def\size{50pt}


    \draw [white,fill = gray]
		(0,0) -- (\size, 0) arc (0:90:\size) -- cycle;
    \draw [white,fill = gray]
		(0,0) -- (-\size, 0) arc (180:270:\size) -- cycle;

    \fill [fill = white] (0, \size) arc (90:270:\size / 5 and \size);
    \fill [fill = white, dashed]
    	(0, -\size) arc (-90:90:\size / 5 and \size);
    \draw [fill = white, dashed,thick]
    	(\size, 0) arc (0:180:50pt and \size / 5);
    \draw [fill = white,thick] (-\size, 0) arc (180:360:50pt and \size / 5);
	\draw [thick] (0, \size) arc (90:270:\size / 5 and \size);
    \draw [dashed,thick] (0, -\size) arc
    	(-90:90:\size / 5 and \size);
    \draw[thick] (0, 0) circle (\size);
\end{tikzpicture}}




\end{document}
