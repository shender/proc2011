\documentclass[11pt]{article}
% math symbols
\usepackage{amssymb,amsmath}
% for different compilers
\usepackage{ifpdf}
% geometry of page
\usepackage[margin=2.1cm]{geometry}
% float pictures
\usepackage{wrapfig}

% if pdflatex, then
\ifpdf
 \usepackage[english,russian]{babel}
 \usepackage[utf8]{inputenc}
 \usepackage[unicode]{hyperref}
 \usepackage[pdftex]{graphicx}
 \usepackage{cmlgc}
% if xelatex, then
\else
% math fonts
 \usepackage{fouriernc}
% xelatex specific packages
 \usepackage[xetex]{hyperref}
 \usepackage{xunicode}	% some extra unicode support
 \usepackage{xltxtra}	% \XeLaTeX macro
 \defaultfontfeatures{Mapping=tex-text}
 \usepackage{polyglossia}	% instead of babel in xelatex
 \setdefaultlanguage{russian}
% fonts
 \setromanfont{Charis SIL}
 \setsansfont{OfficinaSansC} 
 \setmonofont{Consolas}
\fi

% several pictures in one figure
\usepackage{subfig}
% calc in TeX expressions
\usepackage{calc}
% nice pictures and plots
\usepackage{pgfplots,tikz,circuitikz}
% different libraries for pictures
\usetikzlibrary{%
  arrows,%
  calc,%
  patterns,%
  decorations.pathreplacing,%
  decorations.pathmorphing,%
  decorations.markings%
}

% colors of the hyperlinks
\hypersetup{colorlinks,%
  citecolor=blue,%
  urlcolor=blue,%
  linkcolor=red
}

\tolerance=1000
\emergencystretch=0.74cm

\newcommand{\nn}{\nonumber}
\newcommand{\pt}{\partial}
\newcommand{\eps}{\epsilon}
\newcommand{\vareps}{\varepsilon}
\newcommand{\const}{\mathrm{const}}
\newcommand{\com}[1]{{\Large{\texttt{{\color{red}(#1)}}}}}

% счётчик задач
\newcounter{notask}
\setcounter{notask}{1}

% условие без картинки
\newcommand{\task}[1]{
	\hrule
	\hbox to \textwidth {%
    	\vrule
	    \parbox[t]{0.04\textwidth}{\smallskip \centering \arabic{notask}}%
    	\vrule%
	    \hfill%
    	\parbox[t]{0.93\textwidth}{\smallskip #1 \smallskip}\hfill%
	    \vrule
	}
	\hrule
    \addtocounter{notask}{1}
    \pagebreak[2]
}

\newlength{\h}
\newsavebox{\taskbox}
\newlength{\x}
\newsavebox{\pictbox}

% условие с картинкой (картинка выравнивается по центру)
\newcommand{\taskpic}[2]{
	\savebox{\taskbox}{\parbox[t]{0.93\textwidth-4.3cm}{\smallskip #1 \smallskip}}
	\savebox{\pictbox}{\parbox[t]{4cm}{\smallskip \centering
    	\vspace{0pt} #2 \smallskip}}
	\h=\ht\taskbox
	\advance\h\dp\taskbox
	\x=\ht\pictbox
	\advance\x\dp\pictbox
	\hrule
	\hbox to \textwidth {%
		\vrule\parbox[t][\maxof{\h}{\x}][t]{0.04\textwidth}{ \smallskip
    		\centering \arabic{notask} }\vrule%
	    \hfill\parbox[t][\maxof{\h}{\x}][t]{0.93\textwidth-4.3cm}{\smallskip #1
    	    \smallskip}\hfill\vrule%
	    \hfill\parbox[t][\maxof{\h}{\x}][c]{4cm}{\hfil #2 \hfil}\hfill\vrule
	}
	\hrule
	\addtocounter{notask}{1}
	\pagebreak[2]
}

\tikzset{>=latex, interface/.style = {postaction = {draw, decorate,
	    decoration = {border, angle = 45, amplitude = 0.2cm, segment length=1.4323mm}}},%
		spring/.style={decorate,decoration={snake,amplitude=1mm,
                    segment length=2mm},thick},
                marrow/.style={postaction={draw,decorate,decoration={markings,
    mark=at position 0.6 with {\arrow{latex}}}}}}


\begin{document}

\thispagestyle{empty}
\parindent=5mm
\righthyphenmin=2
\begin{center}

\phantom{Бухалкин апанас}

\vfill
\LARGE{\textsc{XVII Летняя Физическая Школа}}\\
\Large{\textsc{15 июля -- 4 августа 2011}}\\[1cm]
\Large{\textit{Сборник материалов}}\\[2cm]
\includegraphics[width=6cm]{logo.pdf}
\vfill

\small{\textsf{Санкт-Петербург}\\
\textsf{2011}}
\end{center}

\clearpage

\hypersetup{colorlinks,%
  linkcolor=black
}

\tableofcontents

\hypersetup{colorlinks,%
  citecolor=blue,
  urlcolor=blue,
  linkcolor=red
}


\clearpage

\section{Участники ЛФШ.}
\label{sec:participants}

\subsection{8 и 9 класс.}
\label{sec:8and9}

\begin{table}[ht]
  \begin{minipage}[t]{0.45\linewidth}\centering
    \begin{center}
      \textit{8 класс}
    \end{center}
    \begin{tabular}[h]{|c|c|c|c|}
      \hline
      \textbf{№} & \textbf{Фамилия, имя} & \textbf{Шк.} & \textbf{Д/б}\\
      \hline
      1 & Артамонов Николай & 239 & 47\\ \hline
      2 & Багиров Фарид & 533 & Д1 \\ \hline
      3 & Баринов Фёдор & 239 & Д3 \\ \hline
      4 & Богомолов Егор &  92 & ПО \\ \hline
      5 & Бомов Фёдор & 292 & 56 \\ \hline
      6 & Галкин Георгий & --- & --- \\ \hline
      7 & Гера Станислава & 534 & Д3 \\ \hline
      8 & Грибакин Борис & 74 & 50 \\ \hline
      9 & Громов Даниил & Ун. & 24 \\ \hline
      10 & Гуменюк Виталий & 92 & 66 \\ \hline
      11 & Жибарев Георгий & 533 & Д3 \\ \hline
      12 & Йона Андрей & 152 & --- \\ \hline
      13 & Кокурушников Тимофей & 239 & Д2 \\ \hline
      14 & Коробов Артём & 2 & 48 \\ \hline
      15 & Кутимский Максим & 239 & --- \\ \hline
      16 & Лебедев Александр & --- & 14 \\ \hline
      17 & Мастеров Роман & 366 & 58 \\ \hline
      18 & Морозов Дмитрий & 239 & Д2 \\ \hline
      19 & Олейник Дарья & 239 & 53 \\ \hline
      20 & Охотников Артём & 239 & Д2 \\ \hline
      21 & Петров Степан & 30 & Д2 \\ \hline
      22 & Приходько Алексей & 366 & ПО \\ \hline
      23 & Родионова Анна & 239 & 54 \\ \hline
      24 & Рутковский Илья & 78 & 46 \\ \hline
      25 & Сокольский Станислав & 366 & 33 \\ \hline
      26 & Староказников Александр & 344 & ПО \\ \hline
      27 & Трофимов Даниил & 101 & Д3 \\ \hline
      28 & Уланова Арина & 533 & Д2 \\ \hline
      29 & Усачёва Мария & 295 & 23 \\ \hline
      30 & Ходунов Павел & 239 & Д1 \\ \hline
      31 & Шубин Григорий & 550 & 42 \\ \hline
    \end{tabular}
  \end{minipage}
  \hfill
  \begin{minipage}[t]{0.45\linewidth}\centering
    \begin{center}
      \textit{9 класс}
    \end{center}
    \begin{tabular}[h]{|c|c|c|c|}
      \hline
      \textbf{№} & \textbf{Фамилия, имя} & \textbf{Шк.} & \textbf{Д/б}\\
      \hline
      1 & Агафонов Игорь & 533 & ПО  \\ \hline
      2 & Андреев Константин & ФТШ & Д2  \\ \hline
      3 & Беляков Михаил & 239 & Д1  \\ \hline
      4 & Блехштейн Максим & 239 & Д3  \\ \hline
      5 & Буренев Иван & ФТШ & Д3  \\ \hline
      6 & Вахренев Роман & 239 & 44  \\ \hline
      7 & Глезеров Евгений & 73 & ---  \\ \hline
      8 & Гордеева Людмила & ФТШ & 16  \\ \hline
      9 & Гуцол Ксения & ФТШ & ПО  \\ \hline
      10 & Затылкин Павел & 239 & 47  \\ \hline
      11 & Затылкин Кирилл & 239 & 40  \\ \hline
      12 & Иванов Владислав & 177 & Д3  \\ \hline
      13 & Киселев Егор & 239 & Д3  \\ \hline
      14 & Коско Софья & 239 & Л  \\ \hline
      15 & Кравченко Дмитрий & --- & 40  \\ \hline
      16 & Куксенок Даниил & 533 & ПО  \\ \hline
      17 & Лихачев Иван & 30 & 44  \\ \hline
      18 & Луцкий Георгий & ФТШ & ПО  \\ \hline
      19 & Малышева Александра & ФТШ & 39  \\ \hline
      20 & Муретова Мария & 239 & Д1  \\ \hline
      21 & Никоненко Михаил & 393 & ---  \\ \hline
      22 & Осипов Игорь & 571 & ---  \\ \hline
      23 & Портянкин Егор & ФТШ & 65  \\ \hline
      24 & Семенов Александр & ФТШ & 31  \\ \hline
      25 & Сычёв Станислав & ФТШ & Д1  \\ \hline
      26 & Хвещук Анастасия & 470 & ---  \\ \hline
      27 & Цейтина Елена & ФТШ & 41  \\ \hline
    \end{tabular}
  \end{minipage}
\end{table}

В \textbf{8 классе} работали: Н.В. Тараканов, М.В. Евтихиев (студент 3-го
курса ФТФ СПбГПУ), А.В. Лиознова (студентка 3-го курса ФТФ СПбГПУ),
И. Авдеев, А. Лиознов (студенты 1-го курса ФТФ СПбГПУ), Д. Максимова
(студентка 1-го курса ФТФ СПбГПУ), С. Богданов (студент 1-го курса ФФ
МГУ).

В \textbf{9 классе} работали: О.В. Шустова (студентка 6-го курса ФТФ СПбГПУ),
Ф. Затылкин (студент 2-го курса ФМФ СПбГПУ), А. Коротченков (студент
2-го курса ФТФ СПбГПУ), Ф. Петухов (студент 1-го курса ФТФ СПбГПУ). 

\clearpage

\subsection{10 и 11 класс.}
\label{sec:10and11}

\begin{table}[ht]
  \begin{minipage}[t]{0.45\linewidth}\centering
    \begin{center}
      \textit{10 класс}
    \end{center}
    \begin{tabular}[h]{|c|c|c|c|}
      \hline
      \textbf{№} & \textbf{Фамилия, имя} & \textbf{Шк.} & \textbf{Д/б}\\
      \hline
      1 & Бальков Андрей & 261 & 37\\ \hline
      2 & Вишняк Сергей & ФТШ & 41 \\ \hline
      3 & Водопьян Даниил & 239 & Р \\ \hline
      4 & Грудкин Антон & 239 & Д1 \\ \hline
      5 & Егоров Антон & ФТШ & Р \\ \hline
      6 & Иващенко Дмитрий & 239 & 32 \\ \hline
      7 & Капралов Николай & ФТШ & 46 \\ \hline
      8 & Конаныхин Роман & ФТШ & Д1 \\ \hline
      9 & Крюков Михаил & ФТШ & 31 \\ \hline
      10 & Лашкевич Злата & ФТШ & 24 \\ \hline
      11 & Люлина Анастасия & ФТШ & 27 \\ \hline
      12 & Максакова Мария & ФТШ & 39 \\ \hline
      13 & Мосягин Иван & 214 & 42 \\ \hline
      14 & Погодаев Илья & 610 & ПО \\ \hline
      15 & Рау Владислава & 239 & 46 \\ \hline
      16 & Серов Юрий & ФТШ & Р \\ \hline
      17 & Томп Дмитрий & ФТШ & Д3 \\ \hline
      18 & Ярковой Алексей & 557 & 24 \\ \hline
    \end{tabular}
  \end{minipage}
  \hfill
  \begin{minipage}[t]{0.45\linewidth}\centering
    \begin{center}
      \textit{11 класс}
    \end{center}
    \begin{tabular}[h]{|c|c|c|c|}
      \hline
      \textbf{№} & \textbf{Фамилия, имя} & \textbf{Шк.} & \textbf{Д/б}\\
      \hline
      1 & Балашов Александр & 30 & ---  \\ \hline
      2 & Борздун Наталья & 239 & ---  \\ \hline
      3 & Грачёв Дмитрий & М & ---  \\ \hline
      4 & Жаровов Дмитрий & 239 & ---  \\ \hline
      5 & Косицын Александр & 239 & ---  \\ \hline
      6 & Максимишин Дмитрий & 239 & ---  \\ \hline
      7 & Маслов Артём & ФТШ & Р  \\ \hline
      8 & Матюшин Георгий & ФТШ & Д1  \\ \hline
      9 & Михайлов Кирилл & 239 & ---  \\ \hline
      10 & Никитин Денис & 239 & Р  \\ \hline
      11 & Свирина Анна & 239 & ---  \\ \hline
      12 & Смирнов Иван & 214 & Д2  \\ \hline
      13 & Терехов Антон & 239 & Р  \\ \hline
      14 & Толстопятов Всеволод & 239 & ---  \\ \hline
      15 & Трофимов Павел & ФТШ & ---  \\ \hline
      16 & Чурилова Мария & ФТШ & ---  \\ \hline
      17 & Шалымов Роман & 239 & Д3  \\ \hline
    \end{tabular}
  \end{minipage}
\end{table}

В \textbf{10 классе} работали: И.А. Барыгин (к.ф.--м.н., преподаватель
ФТШ), С. Атамась, В. Коваленко (студенты 3-го курса ФТФ СПбГПУ).

В \textbf{11 классе} работали: И.Е. Шендерович, Д.О. Соколов
(аспиранты ПОМИ РАН), Д.С. Смирнов (студент 4-го курса ФТФ СПбГПУ). 

\section{Ежедневные занятия.}
\label{sec:daily}

\subsection{8 класс.}
\label{sec:daily8}

\subsubsection{Теория.}
\label{sec:th8}

\textit{Преподаватель: Н.В. Тараканов.}

\subsubsection{Эксперимент.}
\label{sec:exp8}

\textit{Преподаватели: А. Лиознова, Д. Максимова, И. Авдеев.}

\begin{enumerate}
\item Общие понятия о физическом эксперименте. Написание отчёта по
  эксперименту.
\item Составление плана местности в декартовых координатах.\\
  \textit{Оборудование:} линейка, два перпендикулярных друг другу
  ориентира (дорога и забор).
\item По КП, на которых написаны координаты следующей точки в
  декартовых или полярных координатах, пройти трассу.\\
  \textit{Оборудование:} компас, рулетка.
\item Построить и отградуировать на 1 минуту водяные или песочные
  часы. \\
  \textit{Оборудование:} песок, вода, скотч, одноразовые стаканчики,
  бутылка, пластилин, бумага.
\item Измерить скорость муравья.\\
  \textit{Оборудование:} нитка, линейка, секундомер.
\item Измерить ускорение свободного падения. \\
  \textit{Оборудование:} груз известной массы, линейка, секундомер.
\item Измерить трение покоя между линейкой и грузом. \\
  \textit{Оборудование:} камень, две линейки.
\item Исследовать зависимость пути тела от угла бросания. \\
  \textit{Оборудование:} резинка, камень, линейка.
\item Измерить величину центростремительного ускорения.\\
  \textit{Оборудование:} камень, нитка, секундомер.
\item От чего зависит период колебания маятника? \\
  \textit{Оборудование:} грузики, нитки, секундомер.
\end{enumerate}

\subsection{9 класс.}
\label{sec:daily9}

\subsubsection{Теория.}
\label{sec:th9}

\textit{Преподаватель: О.В. Шустова.}

\begin{enumerate}
\item Системы координат. Векторы. Скорость. Масса. Импульс.
\item Равноускоренное движение. Графическое представление.
\item Потенциальные и непотенциальные силы. Работа. Потенциальная
  энергия.
\item Кинетическая энергия. Закон сохранения энергии.
\item Закон сохранения импульса. Абсолютно упругое/неупругое
  соударение.
\item Центр масс. Закон движения центра масс.
\item Движение по окружности. Момент силы, момент импульса.
\item Момент инерции. Расчёт для разных тел.
\item Уравнение моментов. 
\end{enumerate}

\subsubsection{Эксперимент.}
\label{sec:exp9}

\textit{Преподаватели: Ф. Затылкин, Ф. Петухов.}

\subsection{Смешанная группа 8--9 классов.}
\label{sec:daily89}

\subsubsection{Теория.}
\label{sec:th89}

\textit{Преподаватель: Н.В. Тараканов.}

\subsubsection{Эксперимент.}
\label{sec:exp89}

\textit{Преподаватели: С. Богданов, А. Лиознов.}

\subsection{10 класс.}
\label{sec:daily10}

\subsubsection{Теория.}
\label{sec:th10}

\textit{Преподаватель: С. Атамась.}

\subsubsection{Эксперимент.}
\label{sec:exp10}

\textit{Преподаватель: И.А. Барыгин.}\\
\begin{enumerate}
\item Измерить отношение масс двух грузов.\\
  \textit{Оборудование:} два груза, нитки, миллиметровка.
\item ВАХ нелинейного элемента.\\
  \textit{Оборудование:} источник постоянного тока, провода, реостат,
  мультиметр, лампочка.
\item Измерение скорости вытекания воды из крана.\\
  \textit{Оборудование:} линейка.
\item Измерение отношения длин ниток Y-образного маятника.\\
  \textit{Оборудование:} нитки, грузики.
\item Измерить жесткость каучукового шарика.\\
  \textit{Оборудование:} каучуковый шарик, вода, линейка, гуашь.
\item Измерить коэффициент трения линейки по столу.\\
  \textit{Оборудование:} две деревянные линейки.
\item Измерить зависимость мощности теплоотдачи от разницы температур.\\
  \textit{Оборудование:} горячая вода, мерный стакан или линейка,
  термометр для воды, часы, сосуд.
\end{enumerate}

\subsection{11 класс.}
\label{sec:daily11}

\subsubsection{Теория.}
\label{sec:th11}

\textit{Преподаватели: И.Е. Шендерович, Д.О. Соколов.}

\begin{enumerate}
\item Введение. Требующие объяснения экспериментальные факты.
\item Векторный анализ. Циркуляция и поток. Градиент, дивергенция и
  ротор. Теорема Стокса и теорема Гаусса--Остроградского.
\item Электростатика. Теорема Гаусса. Потенциал электростатического
  поля.
\item Переход к движущимся зарядам. Уравнение неразрывности
  электрического заряда, его физический смысл.
\item Уравнение Максвелла для ротора магнитного
  поля. Магнитостатика. Закон Био~--~Савара~--~Лапласа, закон
  Ампера. Примеры: поле кругового тока, поле соленоида, поле длинного
  провода.
\item Ток смещения, его физический смысл. Задача о разрядке
  конденсатора.
\item Электромагнитные волны. Закон индукции Фарадея как необходимое
  условие для их существования. Сила Лоренца. Примеры.
\item Физика индукции. Примеры возникновения ЭДС индукции.
\item Движущееся электромагнитное поле. Скорость волны и скорость
  света.
\item Пример полного решения уравнений Максвелла: переменное поле в
  цилиндрическом конденсаторе. Функция Бесселя.
\item Применение уравнений Максвелла к электрическим цепям. Законы
  Кирхгофа. Примеры расчёта схем.
\item Введение в теорию относительности. Инерциальные системы
  отсчета. Противоречия преобразований Галилея и уравнений Максвелла
  (на основе задачи о пластинах). Постулат об инвариантности скорости
  света. Введение системы аксиом.
\item Интервал. Физический смысл интервала. Интервал, как аналог
  расстояния. Инвариантность интервала.
\item Повороты. Преобразования Лоренца для координат. <<Парадоксы>>
  (неинерциальные системы отсчета, задача о трубе).
\item Преобразования Лоренца для скоростей. Сравнение преобразований
  Лоренца и преобразований Галилея (малые скорости,
  некоммутативность). Простые задачи на преобразование Лоренца.
\item Обзорный рассказ про преобразование для полей. Схема
  строго рассуждения (введение векторного потенциала).
\end{enumerate}

\subsubsection{Эксперимент.}
\label{sec:exp11}

\textit{Преподаватель: Д.С. Смирнов.}\\

Идеи большинства экспериментов взяты из книги С.Д. Варламова,
А.Р. Зильбермана, В.И. Зинковского <<Экспериментальные задачи на
уроках физики и физических олимпиадах>>.

\begin{enumerate}
\item Найти отношение массы монеты к массе листа миллиметровки.\\
  \textit{Оборудование:} монета, миллиметровка.
\item Найти закон распределения времени, отмеряемого внутренними
  часами. Экспериментатор должен сто раз отмерить по внутренним часам
  10 секунд и сравнить результат с истинным.
\item Определить плотность и массу куска пластилина.\\
  \textit{Оборудование:} мерный стакан, миллиметровка.
\item Измерить расстояние между бороздками CD.\\
  \textit{Оборудование:} лазерная указка известной длины волны,
  линейка.
\item Измерить и объяснить ВАХ лампочки накаливания.\\
  \textit{Оборудование:} батарейка 9В, реостат 10КОм, провода, мультиметр. 
\item Определить схему и номиналы элементов в чёрном ящике.\\
  \textit{Оборудование:} мультиметр. 
\item Определить теплоту плавления припоя.\\
  \textit{Оборудование:} свечка, мерный стакан, термопара, вода. 
\item Определить отношение частот всех мод двойного
  маятника.\\
  \textit{Оборудование:} пластилин, нитки, секундомер.
\end{enumerate}

\section{Факультативы.}
\label{sec:spec}

\begin{enumerate}
\item Квантовая механика.\\[0.2cm]
  \textit{Преподаватель: М.В. Евтихиев.}\\[0.2cm]
  История возникновения, формула Планка и фотоэффект. Атом. Модели
  атома, падение заряда на ядро и постулаты Бора. Дифракция и
  интерференция. Опыт по дифракции электронов, волна де
  Бройля. Амплитуда вероятности, принцип неопределенности.
  
\item Статистическая физика. \\[0.2cm]
  \textit{Преподаватель: И.Е. Шендерович.}\\[0.2cm]
  Задачи статистической физики. Вероятности в
  физике. Флуктуации. Связь вероятности флуктуации и минимальной
  работы в термодинамических системах. Распределение газа в закрытом
  сосуде: пример нормального распределения. Понятия среднего и
  среднеквадратичного. Статвес. Энтропия. Распределение
  Гиббса. Распределение Максвелла.
  
\item Явления переноса. \\[0.2cm]
  \textit{Преподаватель: И.А. Барыгин. }\\[0.2cm]
  Явление диффузии. Задача о пьяном матросе. Связь коэффициента
  диффузии с длиной свободного пробега. Уравнение диффузии и его
  функция Грина. Теплопроводность. Температуропроводность. Вязкость,
  динамическая и кинематическая.
\item Радуга. \\[0.2cm]
  \textit{Преподаватель: Д.С. Смирнов.}\\[0.2cm]
  Каков размер радуги? В какой последовательности в ней идут цвета?
  Где небо светлее: внутри радуги или снаружи? Сколько колец радуг
  можно увидеть одновременно? Какая последовательность цветов во всех
  этих кольцах? Поляризована ли радуга?
\item Основы астрофизики. \\[0.2cm]
  \textit{Преподаватель: В. Коваленко.}\\[0.2cm]
  Электромагнитное излучение. Понятие о фотометрии.Ослабление света
  при прохождении сквозь вещество. Понятие спектра. Тепловое
  излучение. Спектр чернотельного излучения. Формула Планка,
  приближения Рэлея--Джинса и Вина. Закон смещения Вина. Спектры
  звезд. Спектральные линии. Эффект Доплера. Смещение спектральных
  линий. Расширение спектральных линий. Спектральные приборы. Методы
  анализа наблюдений. Солнце. Общие сведения. Химический состав и
  спектр. Термоядерные реакции. Нейтринные детекторы на
  Земле. Фотосфера. Общие сведения о звездах. Спектры и
  светимости. Показатель цвета.  Спектральные классы, связь с
  температурой. Абсолютная звездная величина и светимость. Методы
  определения светимости и других характеристик далеких
  звезд. Статистические зависимости между основными
  характеристиками. Диаграмма Герцшпрунга--Рессела. Главная
  последовательность.
\item Тензоры в физике. \\
  \textit{Преподаватель: О.А. Хромов.}
\item Введение в теоретическую механику. \\[0.2cm]
  \textit{Преподаватель: Д.О. Соколов.}\\[0.2cm]
  Введение. Мотивация. Функция Лагранжа. Примеры. Метод
  вариаций. Статика. Форма верёвки в поле тяжести. Движение в поле
  тяжести. Законы сохранения: энергия, импульс, момент импульса.
\end{enumerate}

\section{День Экспериментатора.}
\label{sec:day_exp}

\subsection{8 класс.}
\label{sec:day_exp8}
\begin{enumerate}
\item Высота дерева.\\
  \emph{Оборудование:} линейка.
\item Амплитуда дрожания рук. \\
  \emph{Оборудование:} лазерная указка, миллиметровка.
\item Толщина листа бумаги. \\
  \emph{Оборудование:} лист бумаги, линейка.
\item Длина футбольного поля. \\
  \emph{Оборудование:} линейка.
\item Коэффициент жёсткости резинки. \\
  \emph{Оборудование:} резинка, линейка, груз известной массы, скотч.
\item Плотность пластилина. \\
  \emph{Оборудование:} мерный стакан, стакан с водой, пластилин,
  линейка, груз известной массы.
\end{enumerate}

\subsection{9 класс.}
\label{sec:day_exp9}
\begin{enumerate}
\item Прочность волоса. \\
  \emph{Оборудование:} миллиметровка, груз, линейка, скотч.
\item Диаметр иглы шприца.\\ 
  \emph{Оборудование:} секундомер.
\item Скорость вытекания струи из крана. \\
  \emph{Оборудование:} линейка.
\item Диаметр булавки.\\
  \emph{Оборудование:} фольга, миллиметровка.
\item Сила сжатия прищепки. \\
  \emph{Оборудование:} бумага, грузики, прищепка, нить.
\item Теплоёмкость.\\
  \emph{Оборудование:} вода, груз, тело, термометр.
\item Плотность воды от температуры. \\
  \emph{Оборудование:} термометр, шприц, горячая и холодная вода.
\end{enumerate}

\subsection{10 класс.}
\label{sec:day_exp10}
\begin{enumerate}
\item Измерить силу, необходимую для разрыва волоса.\\
  \emph{Оборудование:} волос, миллиметровка, груз известной массы.
\item Измерить теплоёмкость тела. \\
  \emph{Оборудование:} тело, мерный стакан, горячая вода, термометр.
\item Отношение ёмкостей конденсаторов.\\
  \emph{Оборудование:} мультиметр, батарейка.
\item Измерить внутренний диаметр шприца.\\ 
  \emph{Оборудование:} шприц, линейка, секундомер, вода.
\item Фокусное расстояние линзы. \\
  \emph{Оборудование:} линза, линейка, лазерная указка.
\item ЧЯ (4 резистора квадратиком).\\ 
  \emph{Оборудование:} ЧЯ, мультиметр, батарейка.
\item Момент инерции тела. \\
  \emph{Оборудование:} линейка, миллиметровка, тело, секундомер.
\end{enumerate}

\subsection{11 класс.}
\label{sec:day_exp11}
\begin{enumerate}
\item Скорость вытекания струи из крана. \\
   \emph{Оборудование:} линейка.
\item Отношение ёмкостей. \\
   \emph{Оборудование:} мультиметр.
\item Отношение теплопроводностей двух проволок. \\
   \emph{Оборудование:} проволока, секундомер, воск.
\item Теплоёмкость неизвестного тела. \\
   \emph{Оборудование:} тело, мерный стакан, горячая вода, термометр.
\item Масса бутылки с маслом. \\
   \emph{Оборудование:} волос, верёвка, миллиметровка, груз известной массы.
\item Теплота растворения соли. \\
   \emph{Оборудование:} вода, соль, термометр, мерный стакан.
\item Потеря энергии при столкновении двух монет. \\
   \emph{Оборудование:} миллиметровка, линейка, две монеты.
\end{enumerate}


\clearpage
\section{Материалы физических боёв.}
\label{sec:battles}

\subsection{Физбой 8 класса.}
\label{sec:battle8}

\subsubsection{Полуфинал.}
\label{sec:battle8s}

\taskpic{В цилиндрическом сосуде с водой плавает кусок льда объёмом
  $V_0 = 1000 \text{ см}^3$. В лёд вморожена свинцовая пуля объемом $V
  = 1 \text{ см}^3$. Ко льду на невесомой нерастяжимой нити привязан воздушный
  шарик, заполненный гелием. Оболочка шарика имеет пренебрежимо малую
  массу. Каким должен быть объём шарика $V_1$, чтобы после таяния льда
  уровень воды в сосуде не изменился? Необходимые справочные данные
  приведены ниже:\\
  плотность гелия $\rho_\text{гел} = 0.2 \text{ кг/м}^3$;\\
  плотность воздуха $\rho_\text{возд} = 1.3 \text{ кг/м}^3$;\\
  плотность льда $\rho_\text{льда} = 0.9 \text{ г/см}^3$;\\
  плотность воды $\rho_\text{воды} = 1 \text{ г/см}^3$;\\
  плотность свинца $\rho_\text{свинца} = 11.3 \text{ г/см}^3$.
}{
  \begin{tikzpicture}[scale=1.3]
    \draw[fill=blue!20] (0,0) rectangle (2,1.5);
    \draw[very thick] (0,2) -- (0,0) -- (2,0) -- (2,2);
    \draw[thick,fill=white] (0.7,1.2) rectangle (1.3,1.6);
    \draw[thick] (1,1.6) -- (1,2.5);
    \draw[thick] (1,3) circle (0.5cm) node[blue] {$V_1$};
    \draw[fill=black] (0.9,1.35) circle (0.05cm);
  \end{tikzpicture}
}

\task{ Собака сидит на льду озера, а ее хозяин равномерно удаляется от
  нее со скоростью $v = 2$ м/с. Когда расстояние между собакой и
  хозяином достигает $s = 100$ м, собака решает догнать хозяина, причем
  хочет в момент встречи иметь такую же скорость, как и он. Из-за
  того, что лед скользкий, собака не может развивать ускорение больше
  $a = 2 \text{ м/с}^2$ в каком-либо направлении. За какое минимальное время она
  сможет догнать хозяина?}

\task{ Робот Вася спроектирован так, что может взбираться по
  лестницам. После некоторого времени работы $t$ у Васи садятся
  батарейки. Это время зависит от скорости $v$, с которой Вася движется
  по лестнице. На рисунке приведен график зависимости $1/t(v)$. Какова
  максимальная длина лестницы, на которую может взобраться Вася? Пусть
  теперь Вася пытается взобраться вверх по движущемуся вниз
  эскалатору. Постройте график зависимости максимальной длины
  эскалатора, на который может взобраться Вася, от скорости этого
  эскалатора.}
\vspace{0.5cm}
\begin{center}
  \begin{tikzpicture}
    \begin{axis}[xlabel={$v, \text{ м/с}$},ylabel={$1/t, \text{
          с}^{-1}$},xmin=0,xmax=2,ymin=0,ymax=0.1,minor x tick
      num=5,grid=both,minor y tick num=6,yticklabels={0,,0.02,0.04,0.06,0.08,1},width=10cm]
      \addplot[blue,thick,domain=0:2,samples=50] {0.01+0.015/(2.1-x)};
    \end{axis}
  \end{tikzpicture}
\end{center}

\taskpic{ Имеется система изображенная на рисунке. В бассейне с водой
  находится перевернутый цилиндрический сосуд $F$, закрытый снизу
  невесомым подвижным поршнем $G$. Внутри сосуда вакуум. Блоки $A$,
  $B$, $C$ и $D$ --- неподвижные. Блок $E$ --- подвижный. Вначале
  сосуд держат так, что его дно находится на глубине 5 м. При этом
  поршень находится на глубине 9 м, и все веревки натянуты. Сосуд
  отпускают. На каком расстоянии от уровня воды в бассейне будет
  находиться дно сосуда, когда система придет в положение равновесия?
  Масса сосуда равна 200 кг. Площадь поршня --- 100
  $\text{см}^2$. Плотность воды --- 1000 $\text{кг/м}^3$, атмосферное
  давление --- 100 кПа. Трением пренебречь. Считать сосуд достаточно
  длинным, так, что поршень всегда находиться внутри сосуда, а края
  сосуда всегда находятся под водой.}{
  \begin{tikzpicture}
    \draw[interface,thick] (2,0) -- (1.15,0) (0.85,0) -- (0,0);
    \draw[fill=blue!10] (0.2,2.8) rectangle (1.8,0);
    \draw[very thick] (0.2,3) -- (0.2,0) -- (1.8,0) -- (1.8,3);
    \draw[draw=blue!10,fill=white] (0.8,1.4) rectangle (1.2,2);
    \draw[thick] (0.8,1.2) -- (0.8,2) -- (1.2,2) node[anchor=west]
    {$F$} -- (1.2,1.2);
    \draw[very thick] (0.8,1.4) -- (1.2,1.4) node[right] {$G$};
    \draw[rounded corners=0.15cm] (1,1.4) -- (1,-0.7)
    node[anchor=north] {$D$} -- (2.5,-0.7) node[anchor=north] {$C$} --
    (2.5,2) node[anchor=east] {$E$} (2.8,2)  -- (2.8,1.2);
    \draw[thick,interface] (3,1.2) -- (2.55,1.2);
    \draw (1.15,-0.55) circle (0.15cm);
    \draw[fill=black] (1.15,-0.55) circle (0.01cm);
    \draw (2.35,-0.55) circle (0.15cm);
    \draw[fill=black] (2.35,-0.55) circle (0.01cm);
    \draw (2.65,2) circle (0.15cm);
    \draw[rounded corners=0.15cm] (2.65,2) -- (2.65,4)
    node[anchor=north west] {$B$} -- (1,4) node[anchor=north east] {$A$} --
    (1,2);
    \draw (2.5,3.85) circle (0.15cm);
    \draw[fill=black] (2.5,3.85) circle (0.01cm);
    \draw (1.15,3.85) circle (0.15cm);
    \draw[fill=black] (1.15,3.85) circle (0.01cm);
    % разметка
    \draw[dashed,blue] (0.8,2) -- (-0.2,2);
    \draw[dashed,blue] (0.8,2.8) -- (-0.4,2.8);
    \draw[dashed,blue] (0.8,1.4) -- (-0.4,1.4);
    \draw[blue,<->] (0.1,2) -- (0.1,2.8) node[midway,left] {$h$};
    \draw[blue,<->] (-0.35,1.4) -- (-0.35,2.8) node[midway,left] {$H$};
  \end{tikzpicture}
}

\taskpic{ Два массивных поршня находятся в неподвижной $S$-образной
  жесткой трубке, заполненной водой. К одному из поршней прикреплена
  пружина жесткости $k$ = 1000 Н/м, другой её конец вмонтирован в
  пол. Система находится в равновесии. Правый поршень находится на
  расстоянии $L$ = 20 см от конца трубы. На левый поршень аккуратно
  кладут тяжелый груз. Чему равна максимальная масса груза, при
  которой вода не выливается из системы? Площадь левого поршня равна
  $S_1 = 100 \text{ cм}^2$, площадь правого $S_2 = 500 \text{ cм}^2$,
  постоянная $g$ = 10 Н/кг.}{
  \begin{tikzpicture}
    \draw[very thick,interface] (4,0) -- (0.2,0);
    % жидкость
    \draw[fill=blue!20] (0.9,1) -- (0.5,1) -- (0.5,0) -- (2.5,0) --
    (2.5,2.7) -- (3,2.7) -- (3,2.5) -- (3.5,2.5) -- (3.5,3) -- (2.1,3)
    -- (2.1,0.4) -- (0.9,0.4) -- cycle;
    % сосуд
    \draw[thick] (0.9,3) -- (0.9,0.4) -- (2.1,0.4) -- (2.1,3) --
    (3.5,3) -- (3.5,2);
    \draw[thick] (0.5,3) -- (0.5,0) -- (2.5,0) -- (2.5,2.7) --
    (3,2.7) -- (3,2);
    % поршни
    \draw[fill=black] (0.5,1) rectangle (0.9,1.2)
    node[blue,right] {$S_1$};
    \draw[fill=black] (3,2.5) rectangle (3.5,2.3);
    % разметка
    \draw[blue,<->] (3.7,2.5) -- (3.7,2) node[midway,right] {$L$};
    \draw[blue,->] (2.85,1.5) node[below] {$S_2$} to[out=90,in=200] (2.95,2.4);
    % пружина
    \draw[spring] (3.25,2.3) -- (3.25,0);
  \end{tikzpicture}
}

\task{ Между краями пропасти шириной $H$ = 38 м висит практически
  нерастяжимая веревка длиной $L$ = 40 м. Альпинист массой $m$ = 80 кг
  хочет перебраться по ней через пропасть. Сможет ли он это сделать,
  если веревка рвется при силе натяжения $T$ = 1200 H? Ускорение
  свободного падения $g = 10 \text{ м/с}^2$.}


\clearpage
\subsection{Физбой 9 класса.}
\label{sec:battle9}

\subsubsection{Полуфинал.}
\label{sec:battle9s}

\setcounter{notask}{1}

\task{Космический корабль начинает двигаться прямолинейно с
  ускорением, изменяющимся во времени так, как показано на графике
  (см. рис.). Через какое время корабль удалится от исходной точки в
  положительном направлении на максимальное расстояние? Начальная
  скорость корабля равна нулю.}

\begin{center}
  \begin{tikzpicture}
    \draw[->] (0,0) -- (0,2.5) node[right] {$a, \text{ м/c}^2$};
    \draw[->] (-0.5,1) -- (10.5,1) node[above] {$t$, с};
    \draw[dashed,blue] (0,2) -- (10.5,2);
    \draw[dashed,blue] (0,0.5) -- (10.5,0.5);
    \draw[very thick] (0,2) -- (0.5,2) -- (0.5,0.5) -- (1,0.5) --
    (1,2) -- (1.5,2) -- (1.5,0.5) -- (2.5,0.5) -- (2.5,2) -- (3,2) --
    (3,0.5) -- (4.5,0.5) -- (4.5,2) -- (5,2) -- (5,0.5) -- (7,0.5) --
    (7,2) -- (7.5,2) -- (7.5,0.5) -- (10,0.5);
    \foreach \y in {0.5,1,1.5,2} {
      \draw (-0.1,\y) -- (0.1,\y);
    }
    \foreach \x in {0.5,1,1.5,...,10} {
      \draw (\x,0.9) -- (\x,1.1);
    }
    \draw (-0.3,0.5) node {\small{-1}};
    \draw (-0.3,2) node {\small{2}};
    \draw[white,fill=white] (2.3,0.6) rectangle (2.7,0.9)
    node[midway,black] {\small{5}};
    \draw[white,fill=white] (4.8,0.6) rectangle (5.2,0.9)
    node[midway,black] {\small{10}};
    \draw[white,fill=white] (7.3,0.6) rectangle (7.7,0.9)
    node[midway,black] {\small{15}};
      \end{tikzpicture}
\end{center}

\taskpic{На гладком горизонтальном столе лежат, касаясь друг друга, две
  одинакового размера шайбы 1 и 2, радиус которых равен $R$. Шайбы
  соединены друг с другом с помощью тонкой легкой нити (см. рис., вид
  сверху). Длина нити $L = 2R$. Нить начали тянуть в горизонтальном
  направлении с постоянной силой $F$. Найдите силу, с которой шайбы
  будут давить друг на друга, когда их движение установится. Сила $F$
  приложена в середине нити. Трение можно считать малым. Рассмотрите
  два случая: 1) шайбы имеют одинаковую массу; 2) масса одной шайбы в
  два раза больше массы другой.}{
  \begin{tikzpicture}
    \draw[very thick] (0,0) circle (0.75cm);
    \draw[very thick] (0,1.5) circle (0.75cm);
    \draw[thick] (0,1.5) ++(-20:0.75) -- (1.5,0.75);
    \draw[thick] (0,0) ++(20:0.75) -- (1.5,0.75);
    \draw[very thick,->,blue] (1.5,0.75) -- (2.5,0.75) node[midway,above] {$\vec{F}$};
  \end{tikzpicture}
}

\task{Электрическая цепь составлена из семи последовательно
  соединенных резисторов: $R_1$ = 1 кОм, $R_2$ = 2 кОм, $R_3$ = 3 кОм,
  $R_4$ = 4 кОм, $R_5$ = 5 кОм, $R_6$ = 6 кОм, $R_7$ = 7 кОм и четырех
  перемычек. Входное напряжение U = 53,2 В. Укажите, в каком из
  резисторов сила тока минимальна. Найдите эту силу тока. В каком из
  резисторов сила тока максимальна? Найдите ее.}

\begin{center}
  \begin{circuitikz}
    \draw (0,0) to[generic,l_=$R_1$] (2,0) to[generic,l_=$R_2$] (4,0)
    to[generic,l_=$R_3$] (6,0) to[generic,l_=$R_4$] (8,0)
    to[generic,l_=$R_5$] (10,0) to[generic,l_=$R_6$] (12,0)
    to[generic,l_=$R_7$] (14,0);
    \draw (0,0) -- (0,-1.5) -- (6,-1.5);
    \draw (14,0) -- (14,-1.5) -- (8,-1.5);
    \draw[black,fill=white] (6,-1.5) circle (0.1cm);
    \draw[black,fill=white] (8,-1.5) circle (0.1cm);
    \draw (7,-1.5) node {$U$};
    \draw[thick,looseness=0.5] (0.2,0) to[out=80,in=100] (8,0);
    \draw[thick,looseness=0.5] (4,0) to[out=80,in=100] (12,0);
    \draw[thick,looseness=0.5] (4,0) to[out=80,in=100] (12,0);
    \draw[thick,looseness=0.5] (2,0) to[out=-80,in=-100] (10,0);
    \draw[thick,looseness=0.6] (6,0) to[out=-80,in=-100] (13.8,0);
  \end{circuitikz}
\end{center}

\task{ В сосуде находятся две несмешивающиеся жидкости с удельными
  плотностями $\rho_1$ и $\rho_2$ и толщинами слоёв $h_1$ и $h_2$
  соответственно. С поверхности жидкости в сосуд опускают маленькое
  обтекаемое тело, которое достигает дна как раз в тот момент, когда
  его скорость становится равной нулю. Какова плотность материала, из
  которого сделано тело? }

\clearpage

\task{ При разведении теплолюбивых рыб в аквариуме для поддержания
  необходимой температуры воды $t_\text{т} = 25^{\circ}$C
  используется электрический нагреватель, мощность которого $P_0 =
  100$ Вт. Для хладолюбивых рыб температура воды в аквариуме должна
  быть $t_\text{х} = 12^{\circ}$C. Чтобы обеспечить
  низкотемпературный режим через погруженный в аквариум теплообменник
  - длинную медную трубку - пропускают водопроводную воду, температура
  которой $t_1 = 8^{\circ}$C (эффективность теплообменника столь
  высока, что вытекающая из трубки вода находится в тепловом
  равновесии с водой аквариума). Предполагая, что мощность теплообмена
  между аквариумом и окружающей средой пропорциональна разности
  температур между ними, определите минимальный расход воды
  ($k=\frac{\Delta m}{\Delta \tau}$) для поддержания заданного
  температурного режима. Комнатная температура $t_0 =
  20^{\circ}$C. Удельная теплоемкость воды $c = 4200 \text{ Дж/(кг}
  \cdot \text{К})$. Как изменится ответ, если в аквариуме будут
  разводить рыб, предпочитающих температуру воды $t_\text{х}^*= 16^{\circ}$C?}

\task{ Робот Вася спроектирован так, что может взбираться по
  лестницам. После некоторого времени работы $t$ у Васи садятся
  батарейки. Это время зависит от скорости $v$, с которой Вася движется
  по лестнице. На рисунке приведен график зависимости $1/t(v)$. Какова
  максимальная длина лестницы, на которую может взобраться Вася? Пусть
  теперь Вася пытается взобраться вверх по движущемуся вниз
  эскалатору. Постройте график зависимости максимальной длины
  эскалатора, на который может взобраться Вася, от скорости этого
  эскалатора.}
\vspace{0.5cm}
\begin{center}
  \begin{tikzpicture}
    \begin{axis}[xlabel={$v, \text{ м/с}$},ylabel={$1/t, \text{
          с}^{-1}$},xmin=0,xmax=2,ymin=0,ymax=0.1,minor x tick
      num=5,grid=both,minor y tick num=6,yticklabels={0,,0.02,0.04,0.06,0.08,1},width=10cm]
      \addplot[blue,thick,domain=0:2,samples=50] {0.01+0.015/(2.1-x)};
    \end{axis}
  \end{tikzpicture}
\end{center}

\clearpage
\subsubsection{Финал.}
\label{sec:battle9f}

\setcounter{notask}{1}

\task{Над обрывом установлено орудие, позволяющее вести огонь в любом
  направлении. Снаряды имеют начальную скорость $v$. На расстоянии $l$
  от орудия под углом $\varphi$ к горизонту завис воздушный шар. Известно, что
  шар находится достаточно далеко от орудия --- так, что снаряды в него
  не попадают. Обстрел стали производить снарядами, которые взрываются
  через время $T$ после выстрела. Под каким углом к горизонту следует
  стрелять, чтобы снаряды взрывались как можно ближе к шару? Ускорение
  свободного падения $g$.}

\taskpic{Массивная доска AB скользит со скоростью $u$ по гладкой
  горизонтальной поверхности. Из точки С той же поверхности
  одновременно вылетают две легкие шайбы. Первая шайба скользит по
  поверхности в направлении $CC_1$ параллельно доске АВ со скоростью
  $v_1$, вторая скользит со скоростью $v_2$ под углом $\alpha$ к
  $CC_1$. Через некоторое время шайбы сталкиваются в точке
  D. Определите скорости шайб $v_1$ и $v_2$ до столкновения, если
  известно, что время от начала движения шайб до их столкновения в $n$
  раз превышает время от начала движения шайб до столкновения второй
  шайбы с доской. При ударе шайбы о доску потерь энергии не
  происходит.}{
  \begin{tikzpicture}
    \draw[fill=gray] (0,0) rectangle (0.5,4);
    \draw (0.5,0) node[right] {$A$};
    \draw (0.5,4) node[right] {$B$};
    \draw[thick,blue,->] (0.5,2) -- (1,2) node[midway,above]
    {$\vec{u}$};
    \draw[thick] (3,0) node[right] {$C$} -- (3,4) node[right] {$C_1$};
    \draw[very thick,->,blue] (3,0.5) -- (3,1.5) node[right]
    {$\vec{v}_1$};
    \draw[very thick,->,blue] (3,0.5) -- ++(150:1.2cm) node[above]
    {$\vec{v}_2$};
    \draw[blue] (3,0.85) arc (90:150:0.35cm);
    \draw (2.8,1) node[blue] {$\alpha$};
    \draw[fill=black] (3,3) circle (0.05cm) node[right] {$D$};
  \end{tikzpicture}
}

\task{ Миниатюрный тигель (печка) для плавки металла имеет
  электронагреватель постоянной мощности $P_0 = 20$ Вт. Нагреватель
  включают и, после того как его температура практически перестает
  увеличиваться, в тигель бросают несколько кусочков олова, общая
  масса которых $m = 80$ г. Олово начинает плавиться. График зависимости
  температуры в тигле от времени представлен на рисунке. Определите
  удельную теплоту плавления олова.}
\vspace{0.5cm}
\begin{center}
  \begin{tikzpicture}
    \begin{axis}[xlabel={$\tau, \text{ мин}$},ylabel={$t, {}^{\circ}C$},xmin=0,xmax=32,ymin=0,ymax=320,minor x tick
      num=5,grid=both,minor y tick num=6,width=10cm,smooth]
      \addplot[thick,blue] coordinates {(0,20) (2,130) (4,190) (6,240)
        (8,270) (10,288) (12,300) (14,304) (15,304)};
      \addplot[thick,blue] coordinates { (15,304) (16,237) (17,230)
        (18,230) (20,230) (22,230) (24,230) (26,230) (28,230) (29,237)
      (30,247) (31,260)};
    \end{axis}
  \end{tikzpicture}
\end{center}

\taskpic{Тонкостенный цилиндр катится по горизонтальной поверхности
  без проскальзывания со скоростью $v_0 = 6$ м/с. Коэффициент трения
  между цилиндром и поверхностью равен $\mu= 0{,}2$. Цилиндр сталкивается с
  вертикальной гладкой стенкой и упруго отражается от нее. Определите
  путь, пройденный цилиндром до остановки.}{
  \begin{tikzpicture}
    \draw[thick,interface] (3,2.5) -- (3,0) -- (0,0);
    \draw[very thick] (1,1) circle (1cm);
    \draw[thick,blue,->] (1,1) -- (2.5,1) node[above] {$\vec{v}_0$};
  \end{tikzpicture}
}

\taskpic{ По реке со скоростью $v$ плывут мелкие льдины, которые
  равномерно распределяются по поверхности воды, покрывая ее $n$-ю
  часть. В некотором месте реки образовался затор. В заторе льдины
  полностью покрывают поверхность воды, не нагромождаясь друг на
  друга.  Какая сила действует на 1 м ледяной границы между водой и
  сплошным льдом в заторе со стороны останавливающихся льдин?
  Плотность льда $\rho= 0{,}91 \cdot 10^3 \text{ кг/м}^3$; толщина $h =
  20$ см; скорость реки $v = 0{,}72$ км/ч; плывущие льдины покрывают $n =
  0{,}1$ часть поверхности воды. }{
  \begin{tikzpicture}
    \draw[pattern=north east lines] (0,3) rectangle (4,3.5);
    \draw[pattern=crosshatch] (0,0.5) rectangle (3.5,3);
    \draw[pattern=north east lines] (0,0) rectangle (4,0.5);
    \draw[pattern=north west lines]  (3.5,0.5) rectangle (4,3);
    \draw[very thick,->] (1.5,1.75) -- (2.5,1.75) node[fill=white,midway,above=0.15cm] {$\vec{v}$};
  \end{tikzpicture}
}

\task { Лампа, соединенная последовательно с резистором, сопротивление
  которого $R = 10$ Ом, подключена к сети. Зависимость силы тока от
  напряжения на лампе представлена на рисунке. При каком напряжении
  сети КПД схемы $\eta=25\%$? КПД схемы равен отношению мощности,
  потребляемой лампой, к мощности, потребляемой от сети.}
\vspace{0.5cm}
\begin{center}
  \begin{tikzpicture}
    \begin{axis}[xlabel={$U, \text{ В}$},ylabel={$I, \text{
          A}$},xmin=0,xmax=32,ymin=0,ymax=7,minor x tick
      num=2,grid=both,minor y tick num=2,width=10cm,smooth]
      \addplot[thick,blue] coordinates {(0,0) (5,1.5) (10,4) (15,5.4) (20,6)
        (25,6.2) (30,6.3) };
    \end{axis}
  \end{tikzpicture}
\end{center}

\clearpage
\subsection{Физбой 10 класса.}
\label{sec:battle10}

\setcounter{notask}{1}

\task{С какой силой отталкиваются грани равномерно заряженного по
  поверхности правильного тетраэдра? Длина ребра тетраэдра $a$, заряд
  каждой грани $q$.}

\taskpic{Участок цепи постоянного тока состоит из трех одинаковых
  вольтметров и двух одинаковых амперметров. Показания вольтметров
  V$_1$ и V$_2$ равны $U_1 = 6$~B, $U_2 = 4$~B. Как вы полагаете, что
  показывает третий вольтметр?}{
  \begin{tikzpicture}
    \draw[thick] (0,0.5) -- (0.5,0.5);
    \draw[thick] (0.5,0.5) -- (0.5,2) -- (1.45,2) (2.05,2) -- (2.5,2) --
    (2.5,1) -- (2.7,1);
    \draw[thick] (1.75,2) circle (0.3cm) node[blue] {$V_1$};
    \draw[thick] (3,1) circle (0.3cm) node[blue] {$A_2$};
    \draw[thick] (3.3,1) -- (3.5,1) -- (3.5,0.5) -- (4,0.5);
    \draw[thick] (0.5,1) -- (0.7,1);
    \draw[thick] (1,1) circle (0.3cm) node[blue] {$A_1$};
    \draw[thick] (2.6,1) -- (2.3,1);
    \draw[thick] (2,1) circle (0.3cm) node[blue] {$V_2$};
    \draw[thick] (1.3,1) -- (1.7,1);
    \draw[thick] (0.5,0.5) -- (0.5,0) -- (1.5,0) (2.1,0) -- (3.5,0) --
    (3.5,1);
    \draw[thick] (1.8,0) circle (0.3cm) node[blue] {$V_3$};
  \end{tikzpicture}
}

\task{Ракета пришельцев стартует с поверхности Земли и практически
  мгновенно набирает постоянную вертикальную скорость $v =
  800$~м/с. Злодеи обстреливают ракету из пушки, которая находится в
  $l=40$~км от стартовой площадки. Выстрел производится в момент
  старта. Могут ли злодеи попасть в ракету? Начальная скорость снаряда
  равна $u = 1000$~м/с. Сопротивлением воздуха пренебречь.}

\task{Узкий пучок протонов налетает на шар радиуса $a$. Прицельное
  расстояние (расстояние от центра шара до прямой, на которой лежит
  начальная скорость протонов) $b$, начальная энергия протонов
  $E$. Найти установившийся заряд шара. Заряд протона $e$, масса $m$.}

\taskpic{Кубик массы $M$ стоит на горизонтальной поверхности. Его
  касается кубик массы $m$, висящий на невесомой нерастяжимой
  нити. Нить составляет угол $\alpha$ с вертикалью. В начальный момент
  кубики неподвижны. Определите ускорения кубиков в начальный
  момент. Трением пренебречь. Считайте, что кубики не поворачиваются
  вокруг своей оси.}{
  \begin{tikzpicture}
    \draw[thick,interface] (4,0) -- (0,0);
    \draw[thick] (1.5,0) rectangle (3.5,2) node[midway,blue] {$M$};
    \draw[thick] (0.75,1.25) rectangle (1.5,2) node[midway,blue]
    {$m$};
    \draw (1.12,2) -- (2,4);
    \draw[fill=black] (2,4) circle (0.04cm);
    \draw[dashed] (2,4) -- (2,3);
    \draw[blue] (2,3.25) arc (270:246:0.75);
    \draw[blue] (1.8,3.03) node {$\alpha$};
  \end{tikzpicture}
}


\task{На достаточно удаленные предметы смотрят через собирающую линзу
  с фокусным расстоянием $F=9$~см, располагая глаз на расстоянии
  $a=36$~см от линзы. Оцените минимальный размер экрана, который нужно
  расположить за линзой так, чтобы он перекрыл все поле
  изображения. Считайте, что радиус зрачка равен $r=1{,}5$~мм.}


\clearpage
\subsection{Физбой 11 класса.}
\label{sec:battle11}

\setcounter{notask}{1}
\taskpic{ Два одинаковых кубика с ребром $H$ и массой $2.5m$ каждый стоят
  почти соприкасаясь гранями на гладкой горизонтальной
  поверхности. Сверху на них аккуратно кладут шар массы $m$ и радиуса
  $R$, и он начинает смещаться вертикально вниз, раздвигая кубики в
  стороны. Найти скорость шара непосредственно перед ударом о
  горизонтальную поверхность. Начальная скорость шара пренебрежимо
  мала.  }{
  \begin{tikzpicture}
    \draw[very thick,interface] (4,0) -- (0,0);
    \draw[very thick] (0.5,0) rectangle (1.5,1);
    \draw[very thick] (2.5,0) rectangle (3.5,1);
    \draw[very thick] (2,1.5) circle (0.7cm);
    \draw[blue,<->] (0.75,0) -- (0.75,1) node[midway,right] {$H$};
    \draw[blue,->] (2,1.5) node[above=0.1cm] {$R$} -- ++(40:0.7cm) ; 
  \end{tikzpicture}
}


\taskpic{ За линзой на расстоянии $\ell = 4$ см (больше фокусного)
  расположено перпендикулярно главной оптической оси плоское
  зеркало. Перед линзой, также перпендикулярно главной оптической оси,
  расположен лист клетчатой бумаги. На этом листе получают изображение
  его клеток при двух положениях листа относительно линзы. Эти
  положения отличаются на $L = 9$ см. Определить фокусное расстояние
  линзы.  }{
  \begin{tikzpicture}
    \draw[very thick] (0,0) -- (3.5,0);
    \draw[very thick] (1,1.1) -- (1,-1.1);
    \draw[very thick,<->] (2.5,1.1) -- (2.5,-1.1);
    \draw[very thick,interface] (3.5,1.1) -- (3.5,-1.1);
    \draw[blue,<->] (2.5,-1.2) -- (3.5,-1.2) node[midway,below] {$\ell$};
  \end{tikzpicture}
}

\task{ Вагон массой $M$ и длиной $L$ может без трения двигаться по
  рельсам. Он заполнен газом и разделен пополам подвижной невесомой
  вертикальной перегородкой. Вначале температура газа равна $T$. В
  правой половине включают нагреватель и доводят температуру газа до
  $2T$, в левой части температура остается прежней. Найти перемещение
  вагона, если масса всего газа равна $m$.  }

\task{ Звезда массы $M$ и радиуса $r$ образовалась из однородного
  облака газа с молярной массой $\mu$ радиуса $R$, исходно имевшего
  температуру $T$. Считая звезду также однородной, определите
  среднеквадратичное значение ее угловой скорости.  }

\task{ Радиусы кривизны двух одинаковых, слипшихся друг с другом,
  мыльных пузырей равны $R$. После того, как перегородка лопнула,
  образовался один пузырь радиусом $R_1$. Коэффициент поверхностного
  натяжения мыльного раствора $\sigma$. Найти атмосферное давление.  }

\taskpic{ Имеется равномерно заряженная диэлектрическая
  сфера. Известно, что, если ее разрезать пополам, то <<половинки>>
  будут расталкиваться с силой $F_1$. Если разрезать пополам одну из
  половинок (вдали от второй), то получившиеся <<четвертинки>> будут
  расталкиваться с силой $F_2$. И, наконец, если разрезать пополам
  одну из <<четвертинок>> (вдали от оставшихся частей сферы) на
  <<восьмушки>>, то они будут расталкиваться с силой $F_3$. Найти
  силу, с которой будут расталкиваться <<восьмушки>>, если их
  поместить так, как показано на рисунке.  }{\begin{tikzpicture}

    \def\size{50pt}


    \draw [white,fill = gray]
		(0,0) -- (\size, 0) arc (0:90:\size) -- cycle;
    \draw [white,fill = gray]
		(0,0) -- (-\size, 0) arc (180:270:\size) -- cycle;

    \fill [fill = white] (0, \size) arc (90:270:\size / 5 and \size);
    \fill [fill = white, dashed]
    	(0, -\size) arc (-90:90:\size / 5 and \size);
    \draw [fill = white, dashed,thick]
    	(\size, 0) arc (0:180:50pt and \size / 5);
    \draw [fill = white,thick] (-\size, 0) arc (180:360:50pt and \size / 5);
	\draw [thick] (0, \size) arc (90:270:\size / 5 and \size);
    \draw [dashed,thick] (0, -\size) arc
    	(-90:90:\size / 5 and \size);
    \draw[thick] (0, 0) circle (\size);
\end{tikzpicture}}

\clearpage
\section{Заключительные контрольные.}
\label{sec:final}

\subsection{10 класс.}
\label{sec:final10}

\setcounter{notask}{1}

\task{Четыре одноименных точечных заряда величиной $q$ были
  расположены вдоль одной прямой на расстоянии $r$ друг от
  друга. Какую работу надо совершить, чтобы поместить их в вершинах
  тетраэдра с длиной ребра $r$?}

\task{Металлический шар радиуса 10~см помещен внутрь сферической
  металлической оболочки, имеющей внешний радиус 30~см и толщину
  10~см, так, что их центры совпадают. На шаре находится заряд
  $10^{-5}$~Кл, на оболочке --- заряд $8\cdot10^{-5}$~Кл. Постройте
  графики зависимости напряженности и потенциала электрического поля
  от расстояния до центра шара.}  \taskpic{Пространство между
  обкладками плоского конденсатора заполнено двумя слоями разных
  диэлектриков толщины $d_1$ и $d_2$. Диэлектрическая проницаемость
  диэлектриков $\epsilon_1$ и $\epsilon_2$. Площадь обкладок
  $S$. Найдите емкость конденсатора. Какой заряд будет индуцироваться
  на границе раздела диэлектриков, если на пластинах конденсатора
  разместить заряд $\pm q$?}  {
  \begin{tikzpicture}
    \draw[pattern=north east lines] (0,0) rectangle (3,1);
    \draw[pattern=north west lines] (0,1) rectangle (3,1.5);
    \draw[thick] (0,0) -- (3,0);
    \draw[thick] (0,1.5) -- (3,1.5);
    \draw[thick] (1.5,-0.5) -- (1.5,0);
    \draw[thick] (1.5,1.5) -- (1.5,2);
    \node[anchor=west] at (3,1.25) {$\epsilon_1$};
    \node[anchor=west] at (3,0.5) {$\epsilon_2$};
  \end{tikzpicture}
}  

\taskpic{Найдите количество теплоты, выделившееся на каждом
  сопротивлении после замыкания ключа. Один конденсатор ($C_1$)
  вначале был заряжен до напряжения $V$, а второй не был заряжен.}
{\vspace{0.01mm} \ctikzset {bipoles/length=0.8cm}
  \begin{circuitikz}
    \draw (0,0) 
    to [closing switch] (1,0) 
    to [generic=$R_2$] (3,0) 
    to [C=$C_2$] (3,1.5) 
    to [generic=$R_1$] (0,1.5) 
    to [C=$C_1$] (0,0);
  \end{circuitikz}
}

\taskpic{ Пластины плоского конденсатора имеют форму правильного
  шестиугольника и расположены на расстоянии $d$ друг от друга. Найти
  напряженность поля в точке $A$, расположенной посередине между одной
  из вершин шестиугольника и соответствующей вершиной другого
  шестиугольника. Заряд пластин $q$.}  {
\begin{tikzpicture}
\draw[thick] (0,0) -- (3,0);
\draw[thick] (0,0.5) -- (3,0.5);
\draw[fill=black] (3,0.25) circle (0.02) node[right] {$A$};
\end{tikzpicture}
}
\vspace{0.2cm}
\subsection{11 класс.}
\label{sec:final11}

\setcounter{notask}{1}

\taskpic{Две металлические пластины образуют двугранный угол
  $\pi/4$. На биссектрисе этого угла на расстоянии $h$ от пластин
  находится бесконечный заряженный провод с плотностью заряда
  $\lambda$. Найти силу, с которой провод притягивается к двугранному
  углу.}{
  \begin{tikzpicture}
    \draw[very thick] (2.5,2.5) -- (0,0) -- (3,0);
    \draw[dashed,thick] (0,0) -- (22.5:3cm);
    \draw[thick,blue] (0.8,0) arc (0:45:0.8cm);
    \draw[thick,blue,->] (0.4,1.5) node[left] {$\frac{\pi}{4}$}
    to[out=0,in=30] (0.8,0.5);
    \draw[blue,<->] (22.5:2cm) -- +(-90:.75cm)
    node[midway,right] {$h$};
    \draw[very thick] (0,0) ++ (22.5:2cm) circle (0.1cm);
    \draw[very thick,fill=black] (0,0) ++ (22.5:2cm) circle (0.03cm);
  \end{tikzpicture}}


\taskpic{Длинный прямой провод с током $I$ имеет участок в виде
  полуокружности радиуса $R$. Определите индукцию магнитного поля в
  центре полуокружности.}{
  \begin{tikzpicture}
    \draw[thick] (0,0) -- (1,0) (3,0) -- (4,0);
    \draw[thick,marrow] (1,0) arc (180:0:1cm);
    \draw[blue,thick,->] (2,0) -- ++(30:1cm) node[midway,above] {$R$};
  \end{tikzpicture}}

\taskpic{Виток площади $S$ расположен перпендикулярно магнитному полю
  индукции $B$. Он замкнут через гальванометр с сопротивлением
  $R$. Какой заряд протечёт через гальванометр, если виток повернуть
  параллельно полю?}{
  \begin{tikzpicture}
    \draw[thick] (0,0) ellipse (1.5cm and 0.75cm);
    \draw[thick,dashed] (0,0) ellipse (0.5cm and 1.5cm);
    \draw[thick,->,blue] (0,0) -- ++(0,1cm) node[midway,right] {$\vec{B}$}; 
  \end{tikzpicture}}

\taskpic{Плоский конденсатор помещён в однородное магнитное поле индукции
$B$, параллельное пластинам. Из точки $A$ вылетают электроны в
направлении, перпендикулярном магнитному полю. Напряжение, приложенное
к пластинам, равно $V$. При каком условии электроны будут проходить
через конденсатор?}{
\begin{tikzpicture}
  \draw[line width=0.1cm] (0,1.5) -- (3,1.5);
  \draw[line width=0.1cm] (0,0) -- (3,0);
  \draw[blue,<->] (2.7,0) -- (2.7,1.5) node[midway,right] {$h$};
  \draw[blue,thick,->] (0,0.75) node[above] {$A$} -- (0.5,0.75) node[right] {$v$};
  \draw[blue,<->] (-0.1,0) -- (-0.1,0.75) node[midway,left]
  {$\frac{h}{2}$};
  \draw[thick,blue] (1.8,1) circle (0.1cm) node[right] {$B$};
  \draw[blue,fill=blue] (1.8,1) circle (0.03cm);
  \draw[black,fill=black] (0,0.75) circle (0.04cm);
\end{tikzpicture}}


\end{document}
