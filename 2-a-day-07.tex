\documentclass[12pt]{article}
\synctex=1
% math symbols
\usepackage{amssymb,amsmath}
% for different compilers
\usepackage{ifpdf}
% geometry of page
\usepackage[left=1.8cm,right=1.8cm,top=1cm,bottom=1cm]{geometry}
% float pictures
\usepackage{wrapfig}

% if pdflatex, then
\ifpdf
 \usepackage[english,russian]{babel}
 \usepackage[utf8]{inputenc}
 \usepackage[unicode]{hyperref}
 \usepackage[pdftex]{graphicx}
 \usepackage{cmlgc}
% if xelatex, then
\else
% math fonts
 \usepackage{fouriernc}
% xelatex specific packages
 \usepackage[xetex]{hyperref}
 \usepackage{xunicode}	% some extra unicode support
 \usepackage{xltxtra}	% \XeLaTeX macro
 \defaultfontfeatures{Mapping=tex-text}
 \usepackage{polyglossia}	% instead of babel in xelatex
 \setdefaultlanguage{russian}
% fonts
 \setromanfont{Charis SIL}
 \setsansfont{OfficinaSansC} 
 \setmonofont{Consolas}
\fi

% several pictures in one figure
\usepackage{subfig}
% calc in TeX expressions
\usepackage{calc}
% nice pictures and plots
\usepackage{pgfplots,tikz,circuitikz}
% different libraries for pictures
\usetikzlibrary{%
  arrows,%
  calc,%
  patterns,%
  decorations.pathreplacing,%
  decorations.pathmorphing,%
  decorations.markings%
}
\tikzset{>=latex}

% colors of the hyperlinks
\hypersetup{colorlinks,%
  citecolor=blue,%
  urlcolor=blue,%
  linkcolor=red
}

\tolerance=1000
\emergencystretch=0.74cm

% счётчик задач
\newcounter{notask}
\setcounter{notask}{1}

% условие без картинки
\newcommand{\task}[1]{
  \hrule
  \hbox to \textwidth {%
    \vrule
    \parbox[t]{0.04\textwidth}{\smallskip \centering \arabic{notask}}%
    \vrule%
    \hfill%
    \parbox[t]{0.93\textwidth}{\smallskip #1 \smallskip}\hfill%
    \vrule
  }
  \hrule
  \addtocounter{notask}{1}
  \pagebreak[2]
}

\newlength{\h}
\newsavebox{\taskbox}
\newlength{\x}
\newsavebox{\pictbox}

% условие с картинкой (картинка выравнивается по центру)
\newcommand{\taskpic}[2]{
  \savebox{\taskbox}{\parbox[t]{0.93\textwidth-4.3cm}{\smallskip #1 \smallskip}}
  \savebox{\pictbox}{\parbox[t]{4cm}{\smallskip \centering
      \vspace{0pt} #2 \smallskip}}
  \h=\ht\taskbox
  \advance\h\dp\taskbox
  \x=\ht\pictbox
  \advance\x\dp\pictbox
  \hrule
  \hbox to \textwidth {%
    \vrule\parbox[t][\maxof{\h}{\x}][t]{0.04\textwidth}{ \smallskip
      \centering \arabic{notask} }\vrule%
    \hfill\parbox[t][\maxof{\h}{\x}][t]{0.93\textwidth-4.3cm}{\smallskip #1
        \smallskip}\hfill\vrule%
    \hfill\parbox[t][\maxof{\h}{\x}][c]{4cm}{\hfil #2 \hfil}\hfill\vrule
  }
  \hrule
  \addtocounter{notask}{1}
  \pagebreak[2]
}

\newcommand{\com}[1]{{\Large{\texttt{{\color{red}(#1)}}}}}
% русские единицы измерения в формулах
\newcommand{\unit}[1]{\text{ #1}}
\newcommand{\eps}{\varepsilon}
% не писать номер рисунка в маленьких картинках
\renewcommand{\thesubfigure}{\relax}

% настройки для tikz
% interface = лохматый отрезок
% spring = пружинка
% >=latex --- тип стрелочки
\tikzset{>=latex,%
  interface/.style={postaction={draw,decorate,decoration={border,angle=45,amplitude=0.2cm,segment
        length=1.4323mm}}},%
  spring/.style={decorate,decoration={snake,amplitude=1mm, segment length=2mm},thick}}

\pagestyle{empty}


\begin{document}


\task{Имеется 8~внешне совершенно одинаковых свинцовых шариков, 
однако внутри одного из них сделана небольшая полость. Пользуясь 
только рычажными весами, определите, какой шарик с полостью. Весы 
можно использовать не более двух раз. Опишите свои действия и 
сделайте рисунок.}

\task{Вам даны кастрюля вместимостью 2~л, ведро с водой и чайник, в 
который необходимо как можно точнее отлить из ведра воду объемом 
1~л. Как это можно сделать?}

\vspace{1cm}

\task{Вам дана толстая общая тетрадь и линейка. Определите, как 
можно точнее, толщину тетрадного листа.}

\task{Расстояние от Ленинграда до Москвы 650~км, от Ленинграда до 
станции Бологое --- 280~км. Каково расстояние от Москвы до Бологого, 
если все три населенных пункта лежат на одной прямой?}

\vspace{1cm}

\task{Имеется 8~внешне совершенно одинаковых восковых шаров, однако 
внутрь одного из них закатана небольшая свинцовая дробинка. 
Пользуясь только рычажными весами, определите, в каком шаре 
находится дробинка. Весы можно использовать не более двух раз. 
Опишите свои действия и сделайте рисунок.}

\task{Имеются ведро сухого песка, ведро воды и мензурка. Предложите 
способ нахождения объема пустот в ведре сухого песка.}

\vspace{1cm}

\taskpic{Легкий воздушный шар и пароход движутся прямолинейно 
равномерно параллельными курсами (в одном направлении). Шар 
движется в 2~раза быстрее парохода. Найдите физическую ошибку на 
рисунке. Сделайте верный рисунок и обоснуйте 
его.}{
\begin{tikzpicture}
  \draw (3,3) -- ++(0,-0.75);
  \draw (2.7,2.25) rectangle ++(0.6,-0.4) node[midway] {\tiny{1998}};
  \draw[shading=ball] (3,3) circle (0.3);
  \draw[thick] (0.5,1) -- (1,0.5) -- (2,0.5) -- (2.25,1) -- cycle;
  \draw[thick] (1.25,1) rectangle ++(0.5,0.5);
  \draw[thick] (1.4,1.5) rectangle ++(0.2,0.4);
  \draw[fill=gray] (1.5,1.9) to[out=20,in=200] (2.4,2) to
  [out=80,in=-30] (2.4,2.25) to[out=200,in=20] (2,2.25)
  to[out=180,in=60] (1.5,1.9);
  \draw[very thick,decorate,decoration={random steps, segment
    length=5pt,amplitude=1pt}] (0,0.5) -- ++(3,0);
  \draw (0.9,0.9) node {\tiny{нос}};
\end{tikzpicture}  
}

\task{Три лучника стоят на одной линии на расстоянии 2~м друг от 
друга и стреляют последовательно через 1~с перпендикулярно линии 
стрельбы по длинной мишени, которая движется параллельно линии 
стрельбы со скоростью 0{,}5~м/с. На каком расстоянии друг от друга 
попадут стрелы в мишень, если известно, что все они летят с 
одинаковыми скоростями?}

\vspace{1cm}

\taskpic{В залив впадают два ручья. Можно ли на лодке двигаться по 
заливу так, чтобы, удаляясь от устья одного ручья, оставаться на 
постоянном расстоянии от устья другого? Ответ обоснуйте и 
сопроводите рисунком. Первоначальное расположение лодки по 
отношению к ручьям приведено на рисунке.}{
\begin{tikzpicture}
  \draw[fill=blue!20,draw=blue!20] (0,1.5) rectangle (3,0.5);
  \draw[very thick,decorate,decoration={random steps, segment
    length=5pt,amplitude=1pt}] (0,1.5) -- ++(3,0);
  \draw[blue,thick,decorate,decoration={random steps, segment
    length=5pt,amplitude=3pt}] (1,3.5) -- ++(0,-2);
  \draw[blue,thick,decorate,decoration={random steps, segment
    length=5pt,amplitude=3pt}] (2,3.5) -- ++(0,-2);
  \draw[thick] (2.15,1.25) -- ++(0.25,-0.25);
  \draw[thick] (2.4,1.25) -- ++(-0.25,-0.25);
\end{tikzpicture}  
}

\task{Плот плывет по течению реки в южном направлении. На плоту 
сидит мальчик и ловит рыбу. В какую сторону от положения 
вертикали будет отклоняться леска (и будет ли) находящаяся под 
водой?}

\vspace{1cm}

\task{В тире четыре стрелка, стоя на одной линии, стреляют в 
движущуюся мишень. На каком расстоянии будут следы от пуль на 
мишени, если она движется со скоростью 0{,}5~м/с параллельно линии 
стрельбы; стрелки стоят на расстоянии 1~м друг от друга и стреляют 
последовательно через 1~с?}

\task{Воздушный шар уносится ветром в северо-восточном 
направлении. В какую сторону будут развеваться флаги, украшающие 
шар?}

\vspace{1cm}

\task{Трем туристам необходимо за максимально короткий срок 
добраться из одного населенного пункта в другой. Как это сделать, 
если у них имеется лишь один одноместный велосипед? Скорость 
пешехода в 2 раза меньше велосипедиста. Построить график 
зависимость пройденного пути от времени для всех тел, 
участвующих в движении.}

\task{У вас есть моток тонкой проволоки, карандаш и тетрадь в 
клетку. Как можно определить диаметр поперечного сечения 
проволоки?}

\vspace{1cm}

\task{Имеются ведро сухого песка, ведро воды и мензурка. Предложите 
способ нахождения объема пустот в ведре сухого песка.}

\task{За сутки молодой бамбук может вырасти на 86,4 см. На сколько он 
вырастает за секунду?}

\vspace{1cm}

\task{На какой угол поворачивается Земля вокруг своей оси за 1~мин?}

\task{Сколько потребовалось бы времени для того, чтобы уложить в 
ряд кубики, объемом 1~мм$^3$ каждый, взятые в таком количестве, 
сколько содержится их в 1~м$^3$ , если на укладку одного кубика 
затрачивается время, равное 1~с?}

\vspace{1cm}

\task{Четыре мальчика отправились из одного населенного пункта в 
другой, имея лишь один одноместный велосипед. Известно, что 
скорость велосипедиста в 4~раза больше скорости пешехода. Что 
необходимо сделать, чтобы как можно быстрее проделать этот путь? 
Построить график зависимости пройденного пути от времени для 
всех тел, участвующих в движении.}

\task{Мальчик идет из дома в школу, расстояние от дома до которой 
1200~м. Он всегда приходил в школу в одно и то же время. Но на трети 
пути он вспоминает, что забыл дневник, и решает вернуться домой 
за дневником. С какой скоростью он должен бежать с этого момента, 
чтобы успеть в школу в то же время, если обычно он идет с 
постоянной скоростью 7{,}2~км/ч?}

\clearpage

\task{Имеются ведро сухого песка, ведро воды и мензурка. Предложите 
способ нахождения собственного объема песчинок в ведре сухого 
песка.}

\task{Во время археологических раскопок была найдена старинная 
бутылка, нижняя часть которой имеет форму параллелепипеда и по 
объему составляет около $\frac{2}{3}$ от всей бутылки. Верхняя часть 
бутылки имеет неправильную форму. Имея в распоряжении линейку, 
пробку к этой бутылке и неограниченные запасы воды, определите 
объем этой бутылки.}

\vspace{1cm}

\task{Почему при резком торможении передним колесом велосипеда 
есть опасность перелететь через руль?}

\taskpic{Вода вытекает с постоянной скоростью $v$ из двух одинаковых 
труб в одну трубу с площадью поперечного сечения в 3~раза большей, 
чем у каждой из двух предыдущих труб. Какова скорость $u$ течения 
воды в трубе большого сечения? Вода полностью заполняет 
трубы.}{
\begin{tikzpicture}
  \draw[thick] (0.5,3.5) -- ++(1.5,-1) -- ++(1.5,0);
  \draw[thick] (0.5,0.5) -- ++(1.5,1) -- ++(1.5,0);
  \draw[thick,rounded corners=10] (0.5,3) -- (2,2) -- (0.5,1);
  \begin{scope}[rotate around={atan(2/3):(1.25,1)}]
    \draw (1.25,1) -- ++(0,0.4);
    \draw[->] (1.25,1.2) -- ++(0.5,0) node[right=-3] {\tiny{$v$}};
  \end{scope}
  \begin{scope}[rotate around={-atan(2/3):(1.25,3)}]
    \draw (1.25,3) -- ++(0,-0.4);
    \draw[->] (1.25,2.8) -- ++(0.5,0) node[right=-3] {\tiny{$v$}};
  \end{scope}
  \draw (2.5,1.5) -- ++(0,1);
  \draw[->] (2.5,2) -- ++(0.75,0) node[right] {$u$};
\end{tikzpicture}  
}

\vspace{1cm}

\task{Имеются рычажные весы с чашами различной массы, набор 
одинаковых кубиков, набор одинаковых шариков. Весы находятся в 
равновесии, если положить: на левую чашу 2~кубика, на правую 
3~шарика; или на левую чашу 1~шарик, на правую 1~кубик. Какая чаша 
весов опустится, если положить: на левую чашу 1~кубик, на правую 
1~шарик? Ответ обоснуйте.}

\task{На одной из линий метрополитена все станции расположены на 
одинаковом расстоянии. На преодоление этого расстояния поезд 
метро тратит 4~минуты. В каждую сторону поезда ходят один раз в 
3~минуты. На одной из станций машинист заметил, что на 
противоположной платформе стоит поезд. Через какое время он 
снова увидит поезд, следующий в противоположную сторону?}

\vspace{1cm}

\task{На линии метро расположено 10~станций на одинаковом 
расстоянии друг от друга. Между соседними станциями поезд 
движется 3~минуты. Линию обслуживают 18~поездов. К очередному 
празднику было решено ввести новую конечную станцию. Её 
расположили так, что время движения от неё до ближайшей станции 
метро составило 6~минут. На всех станциях поезд проводит 3~минуты. 
На конечных станциях поезд также стоит 3~минуты, после чего едет в 
обратном направлении. Сколько нужно ввести дополнительных 
поездов, чтобы интервал между их появлениями (в одном 
направлении) на станциях остался прежним?}

\task{С территории военной части Х, расположенной вблизи города Y, 
одновременно выехали три танка. Ехали они по одной дороге, и 
скорость каждого из них была постоянна. Скорость первого танка 
равнялась $v_1 = 30\mbox{ км/ч}$, скорость второго $v_2 = 20\mbox{ км/ч}$. Первый 
танк въехал в город Y в 19{.}00, второй танк --- в 20{.}00, а третий танк --- в 
21{.}00. Найдите скорость третьего танка.}

\end{document}
